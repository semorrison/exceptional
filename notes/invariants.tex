\documentclass[12pt]{amsart}
\usepackage{amssymb}
\usepackage{cite}
\usepackage{array}
\usepackage{booktabs}
\usepackage{mdwtab}
\usepackage{mathtools}
\usepackage[T1]{fontenc}
\usepackage[utf8]{inputenc}
\usepackage{hyphenat}
\usepackage{enumitem}
\usepackage{ifpdf}
\ifpdf
  \usepackage[pdftex]{graphicx}
  \usepackage[pdftex,margin=1.25in]{geometry}
  \usepackage[bookmarks=true, bookmarksopen=true,%
    bookmarksdepth=3,bookmarksopenlevel=2,%
    colorlinks=true,%
    linkcolor=blue,%
    citecolor=blue,%
    filecolor=blue,%
    menucolor=blue,%
    urlcolor=blue]{hyperref}
\else
  \usepackage[dvips]{graphicx}
  \usepackage[dvips,margin=1in]{geometry}
  % Use hyperref with all features turned off even in DVI mode, since
  % the .aux file format changes
  \usepackage[draft]{hyperref}
\fi
\usepackage{url}
\usepackage{todonotes}

% A binary operator with a subscript on both sides (and correct spacing)
% Name stands for subscript-operator-subscript
\newcommand{\sos}[3]{\mathbin{{}_{#1}\mathord#2_{#3}}}

% manyindices
% Adapted from code by "bza" in comp.text.tex, Feb. 7, 2006
%% USAGE:
%%
%% \manyindices#1#2#3#4#5
%%
%% #1=lower left index
%% #2=upper left index
%% #3=lower right index
%% #4=upper right index
%% #5=main symbol
\makeatletter
\newcommand\mi@kern[1]{%
  \settowidth\@tempdima{$\mi@obj^{#1}$}
  \kern-\@tempdima
  #1
  \settowidth\@tempdima{$\mi@obj$}
  \kern\@tempdima
}

\newtoks\mi@toksp
\newtoks\mi@toksb
\DeclareRobustCommand{\manyindices}[5]{
  \def\mi@obj{#5}
  \mi@toksp\expandafter{\mi@kern{#2}}
  \mi@toksb\expandafter{\mi@kern{#1}}
  \@mathmeasure4\textstyle{#5_{#1}^{#2}}
  \@mathmeasure6\textstyle{#5_{#3}^{#4}}
  \dimen0-\wd6 \advance\dimen0\wd4
  \@mathmeasure8\textstyle{\hphantom{{}_{#1}^{#2}}#5^{\the\mi@toksp#4}_{\the\mi@toksb#3}}
  \hbox to \dimen0{}{\kern-\dimen0\box8}
}
\makeatother 

% Left sub/super scripts
% \lsup is a temporary definition until something better is worked out
% Use \lsupv if the next argument is vertical
\newcommand{\lsub}[2]{{}_{#1}#2}
\newcommand{\lsup}[2]{{}^{#1}\mskip-.6\thinmuskip#2}
\newcommand{\lsupv}[2]{{}^{#1}#2}
\newcommand{\lsubsup}[3]{\manyindices{#1}{\mskip.6\thinmuskip#2\mskip-.6\thinmuskip}{}{}{\mathord{#3}}}
\newcommand{\lsubsupv}[3]{\manyindices{#1}{\mskip.2\thinmuskip#2\mskip-.2\thinmuskip}{}{}{\mathord{#3}}}

\newcounter{saveenum}

% Read the file, if it exists
\newread\testin
\def\maybeinput#1{
\openin\testin=#1
\ifeof\testin\typeout{Warning: input #1 not found}\else\input#1\fi
\closein\testin
}

\def\mathcenter#1{%
  \vcenter{\hbox{$#1$}}%
}

\def\graph#1{
        \includegraphics{#1}
}

\def\mathgraph#1{
        \mathcenter{\graph{#1}}
}

\def\mfig#1{
        \mathcenter{\includegraphics{#1}}
}

\def\mfigb#1{
        \mathcenter{\includegraphics[trim=-1 -1 -1 -1]{#1}}
}


%%% Local Variables: 
%%% mode: latex
%%% TeX-master: "main"
%%% End: 

% General use
\newcommand{\RR}{\mathbb R}
\newcommand{\CC}{\mathbb C}
\newcommand{\ZZ}{\mathbb Z}
\newcommand{\QQ}{\mathbb Q}
\newcommand{\PP}{\mathbb P}
\newcommand{\EE}{\mathbb E}
\newcommand{\HH}{\mathbb H}
\newcommand{\NN}{\mathbb N}

\newcommand{\comma}{\mathbin ,}
\newcommand{\conn}{\mathbin \#}
\newcommand{\sltwo}{{{\mathfrak{sl}}_2}}
\renewcommand{\sl}{\mathfrak{sl}}
\newcommand{\gl}{\mathfrak{gl}}
\newcommand{\fg}{{\mathfrak g}}
\newcommand{\co}{\colon\thinspace}
\newcommand{\eps}{\varepsilon}
\newcommand{\abs}[1]{{\lvert #1 \rvert}}
\newcommand{\norm}[1]{{\lVert #1 \rVert}}
\newcommand{\OneHalf}{{\textstyle\frac{1}{2}}}

% Synonyms for commands I never remember
\newcommand{\isom}{\cong}
\newcommand{\superset}{\supset}
\newcommand{\bigcircle}{\bigcirc}
\newcommand{\contains}{\ni}
\newcommand{\tensor}{\otimes}
\newcommand{\bdy}{\partial}

% Stupid overloading.
\newcommand{\lbracket}{[}
\newcommand{\rbracket}{]}

% Various operators.
\DeclareMathOperator{\ad}{ad}
\DeclareMathOperator{\Ad}{Ad}
\DeclareMathOperator{\End}{End}
\DeclareMathOperator{\sign}{sign}
\DeclareMathOperator{\Sym}{Sym}
\DeclareMathOperator{\tr}{tr}
\DeclareMathOperator{\Hom}{Hom}
\DeclareMathOperator{\vol}{vol}
\DeclareMathOperator{\rank}{rank}
\DeclareMathOperator{\im}{im}

% Linear groups
\DeclareMathOperator{\ISO}{\mathit{ISO}}
\DeclareMathOperator{\SO}{\mathit{SO}}
\DeclareMathOperator{\GL}{\mathit{GL}}
\DeclareMathOperator{\SL}{\mathit{SL}}
\DeclareMathOperator{\PSL}{\mathit{PSL}}

% citations
\newcommand{\arxiv}[1]{\href{http://arxiv.org/abs/#1}{\tt arXiv:\nolinkurl{#1}}}
\newcommand{\doi}[1]{\href{http://dx.doi.org/#1}{{\tt DOI:#1}}}
\newcommand{\euclid}[1]{\href{http://projecteuclid.org/getRecord?id=#1}{{\tt #1}}}
\newcommand{\mathscinet}[1]{\href{http://www.ams.org/mathscinet-getitem?mr=#1}{\tt #1}}
\newcommand{\googlebooks}[1]{(preview at \href{http://books.google.com/books?id=#1}{google books})}
\renewcommand{\googlebooks}[1]{}
\newcommand{\numdam}[1]{\href{http://www.numdam.org/item?id=#1}{\tt #1}}

% Theorems
\theoremstyle{plain}
\newtheorem{theorem}{Theorem}
\newtheorem{proposition}{Proposition}
\numberwithin{proposition}{section}
\newtheorem{lemma}[proposition]{Lemma}
\newtheorem{corollary}[proposition]{Corollary}
\newtheorem{claim}[proposition]{Claim}
\newtheorem{conjecture}[proposition]{Conjecture}
\newtheorem{observation}[proposition]{Observation}

\theoremstyle{definition}
\newtheorem{definition}[proposition]{Definition}
\newtheorem{exercise}[proposition]{Exercise}
\newtheorem{question}[proposition]{Question}
\newtheorem{problem}[proposition]{Problem}

\theoremstyle{remark}
\newtheorem{example}[proposition]{Example}
%\newtheorem{hint}[proposition]{Hint}
\newtheorem{remark}[proposition]{Remark}
%\newtheorem{apology}[proposition]{Apology}
%\newtheorem{warning}[proposition]{Warning}

% Hyphenation.
\hyphenation{Thurs-ton}

%ToDoNotes:
\newcommand{\nn}[1]{{\color{red}[[#1]]}}
\newcommand{\DPTtodo}[1]{\todo[color=green!40]{#1}}
\newcommand{\NStodo}[1]{\todo[color=blue!40]{#1}}
\newcommand{\SMtodo}[1]{\todo[color=red!40]{#1}}
\newcommand{\citationneeded}{\ \parbox{1.25in}{\todo[inline]{citation needed}}\ }
\newcommand{\referenceneeded}{\ \parbox{1.35in}{\todo[inline]{reference needed}}\ }

% Commands for exceptional paper
\newcommand{\Sk}[1]{\mathop{\mathrm{Sk}}(#1)}
\newcommand{\Skq}[1]{\mathop{\mathrm{Sk}_q}(#1)}
\newcommand{\Skcat}{\mathop{\mathsf{Sk}}}
\newcommand{\Skqcat}{\mathop{\mathsf{Sk}_q}}
\DeclareMathOperator{\eval}{eval}

\DeclareMathOperator{\Tw}{Tw}
\DeclareMathOperator{\HTw}{HTw}
\DeclareMathOperator{\Fr}{Fr}

\DeclareMathOperator{\fork}{fork}
\DeclareMathOperator{\fuse}{fuse}


%%% Local Variables: 
%%% mode: latex
%%% TeX-master: "main"
%%% TeX-master: t
%%% End: 

\DeclareMathOperator{\Tw}{Tw}
\DeclareMathOperator{\HTw}{HTw}
\DeclareMathOperator{\Fr}{Fr}

\begin{document}
\title{Link invariants from the quantum exceptional series}

\author[Morrison]{Scott Morrison}
\address{Mathematical Sciences Institute, Australian National University}
\email{scott.morrison@anu.edu.au}

\author[Snyder]{Noah Snyder}
\address{Bloomington, Indiana, USA}
\email{nsnyder1@indiana.edu} % Noah, do you prefer a different e-mail address?

\author[Thurston]{Dylan~P.~Thurston}
\address{Bloomington, Indiana, USA}
\email{dpthurst@indiana.edu}

\maketitle

\newcommand{\V}{\mathcal{P}}
\newcommand{\nn}[1]{{\color{red}[[#1]]}}

\section{Link invariants}
\subsection{Matrices}
Recall that we have candidate bases for $\V_n = \operatorname{Hom}(1 \to V^{\otimes n})$, for $n$ at most 6.
In this section, we obtain matrices for the operators braiding and rotation operators on $\V_n$.
\nn{diagrams defining these as the rotation and the braiding}.

We have a pairing $\langle , \rangle: \V_n \otimes \V \to \V_0$ defined by \nn{diagram}, and for many 
(conjecturally all) diagrams in $\V_0$ we can use the relations to evaluate to a rational function 
in $v$ and $w$. Of course, our basis is far from orthonormal with respect to this pairing!
We find, however, that up to $n=6$, we can evaluate all matrix entries of
$M = \left(\langle e_i, e_j \rangle\right)_{i,j}$,
but rather unfortunately inverting this matrix is computationally infeasible \nn{for which n?}.
(Recall that arithmetic over $\mathcal{Q}(v,w)$ is inefficient.)

Now, given an operator $T: \V_n \to \V_n$, the associated matrix with respect to our chosen basis is
$[T] = AM^{-1}$, where $A = \left(\langle T e_i, e_j \rangle\right)_{i,j}$. Again, we find that for
$T$ either the braiding or the rotation, we can evaluate all of the matrix entries of the $A$ 
as rational functions. Even though directly inverting $M$ is infeasible, we find that we can
calculate the product $A M^{-1}$ in each case.
(Nevertheless this remains a difficult calculation. \nn{some explanation of what we do, with the CRT})

\nn{Explicit matrices are available with the {\tt arXiv} sources of this article.}

\subsection{Limited planar operations}
Thinking of the vector space $\V_n$ as forming a planar algebra, 
at this point we have limited access to the full set of operations indexed. We can compute
$cap : \V_n \to \V_{n-2}$, 
$cup : \V_n \to V_{n+2}$, 
$\rho: \V_n \to \V_n$, 
$m_{4,4,2} : \V_4 \otimes \V_4 \to \V_4$,
$m_{6,4,2} : \V_6 \otimes \V_4 \to \V_6$, 
and finally $m_{4,4,1} : \V_4 \otimes \V_4 \to V_6$.

\nn{explain how we actually compute these maps, from above!}


\subsection{6-box presentations of links}
Typically, we may choose to think about a diagrammatically defined tangle invariant as a 
morphism of planar algebras $\operatorname{Tangle} \to \mathcal{Q}$ for some planar algebra 
$\mathcal{Q}$,
sending the crossing to a suitable element of $\mathcal{Q}_4$. In our case only know a 
limited set of operations in the target planar algebra


\end{document}
