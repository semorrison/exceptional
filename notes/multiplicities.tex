\documentclass[12pt]{amsart}
\usepackage{amssymb}
\usepackage{cite}
\usepackage{array}
\usepackage{booktabs}
\usepackage{mdwtab}
\usepackage{mathtools}
\usepackage[T1]{fontenc}
\usepackage[utf8]{inputenc}
\usepackage{hyphenat}
\usepackage{enumitem}
\usepackage{ifpdf}
\ifpdf
  \usepackage[pdftex]{graphicx}
  \usepackage[pdftex,margin=1.25in]{geometry}
  \usepackage[bookmarks=true, bookmarksopen=true,%
    bookmarksdepth=3,bookmarksopenlevel=2,%
    colorlinks=true,%
    linkcolor=blue,%
    citecolor=blue,%
    filecolor=blue,%
    menucolor=blue,%
    urlcolor=blue]{hyperref}
\else
  \usepackage[dvips]{graphicx}
  \usepackage[dvips,margin=1in]{geometry}
  % Use hyperref with all features turned off even in DVI mode, since
  % the .aux file format changes
  \usepackage[draft]{hyperref}
\fi
\usepackage{url}
\usepackage{todonotes}

% A binary operator with a subscript on both sides (and correct spacing)
% Name stands for subscript-operator-subscript
\newcommand{\sos}[3]{\mathbin{{}_{#1}\mathord#2_{#3}}}

% manyindices
% Adapted from code by "bza" in comp.text.tex, Feb. 7, 2006
%% USAGE:
%%
%% \manyindices#1#2#3#4#5
%%
%% #1=lower left index
%% #2=upper left index
%% #3=lower right index
%% #4=upper right index
%% #5=main symbol
\makeatletter
\newcommand\mi@kern[1]{%
  \settowidth\@tempdima{$\mi@obj^{#1}$}
  \kern-\@tempdima
  #1
  \settowidth\@tempdima{$\mi@obj$}
  \kern\@tempdima
}

\newtoks\mi@toksp
\newtoks\mi@toksb
\DeclareRobustCommand{\manyindices}[5]{
  \def\mi@obj{#5}
  \mi@toksp\expandafter{\mi@kern{#2}}
  \mi@toksb\expandafter{\mi@kern{#1}}
  \@mathmeasure4\textstyle{#5_{#1}^{#2}}
  \@mathmeasure6\textstyle{#5_{#3}^{#4}}
  \dimen0-\wd6 \advance\dimen0\wd4
  \@mathmeasure8\textstyle{\hphantom{{}_{#1}^{#2}}#5^{\the\mi@toksp#4}_{\the\mi@toksb#3}}
  \hbox to \dimen0{}{\kern-\dimen0\box8}
}
\makeatother 

% Left sub/super scripts
% \lsup is a temporary definition until something better is worked out
% Use \lsupv if the next argument is vertical
\newcommand{\lsub}[2]{{}_{#1}#2}
\newcommand{\lsup}[2]{{}^{#1}\mskip-.6\thinmuskip#2}
\newcommand{\lsupv}[2]{{}^{#1}#2}
\newcommand{\lsubsup}[3]{\manyindices{#1}{\mskip.6\thinmuskip#2\mskip-.6\thinmuskip}{}{}{\mathord{#3}}}
\newcommand{\lsubsupv}[3]{\manyindices{#1}{\mskip.2\thinmuskip#2\mskip-.2\thinmuskip}{}{}{\mathord{#3}}}

\newcounter{saveenum}

% Read the file, if it exists
\newread\testin
\def\maybeinput#1{
\openin\testin=#1
\ifeof\testin\typeout{Warning: input #1 not found}\else\input#1\fi
\closein\testin
}

\def\mathcenter#1{%
  \vcenter{\hbox{$#1$}}%
}

\def\graph#1{
        \includegraphics{#1}
}

\def\mathgraph#1{
        \mathcenter{\graph{#1}}
}

\def\mfig#1{
        \mathcenter{\includegraphics{#1}}
}

\def\mfigb#1{
        \mathcenter{\includegraphics[trim=-1 -1 -1 -1]{#1}}
}


%%% Local Variables: 
%%% mode: latex
%%% TeX-master: "main"
%%% End: 

% General use
\newcommand{\RR}{\mathbb R}
\newcommand{\CC}{\mathbb C}
\newcommand{\ZZ}{\mathbb Z}
\newcommand{\QQ}{\mathbb Q}
\newcommand{\PP}{\mathbb P}
\newcommand{\EE}{\mathbb E}
\newcommand{\HH}{\mathbb H}
\newcommand{\NN}{\mathbb N}

\newcommand{\comma}{\mathbin ,}
\newcommand{\conn}{\mathbin \#}
\newcommand{\sltwo}{{{\mathfrak{sl}}_2}}
\renewcommand{\sl}{\mathfrak{sl}}
\newcommand{\gl}{\mathfrak{gl}}
\newcommand{\fg}{{\mathfrak g}}
\newcommand{\co}{\colon\thinspace}
\newcommand{\eps}{\varepsilon}
\newcommand{\abs}[1]{{\lvert #1 \rvert}}
\newcommand{\norm}[1]{{\lVert #1 \rVert}}
\newcommand{\OneHalf}{{\textstyle\frac{1}{2}}}

% Synonyms for commands I never remember
\newcommand{\isom}{\cong}
\newcommand{\superset}{\supset}
\newcommand{\bigcircle}{\bigcirc}
\newcommand{\contains}{\ni}
\newcommand{\tensor}{\otimes}
\newcommand{\bdy}{\partial}

% Stupid overloading.
\newcommand{\lbracket}{[}
\newcommand{\rbracket}{]}

% Various operators.
\DeclareMathOperator{\ad}{ad}
\DeclareMathOperator{\Ad}{Ad}
\DeclareMathOperator{\End}{End}
\DeclareMathOperator{\sign}{sign}
\DeclareMathOperator{\Sym}{Sym}
\DeclareMathOperator{\tr}{tr}
\DeclareMathOperator{\Hom}{Hom}
\DeclareMathOperator{\vol}{vol}
\DeclareMathOperator{\rank}{rank}
\DeclareMathOperator{\im}{im}

% Linear groups
\DeclareMathOperator{\ISO}{\mathit{ISO}}
\DeclareMathOperator{\SO}{\mathit{SO}}
\DeclareMathOperator{\GL}{\mathit{GL}}
\DeclareMathOperator{\SL}{\mathit{SL}}
\DeclareMathOperator{\PSL}{\mathit{PSL}}

% citations
\newcommand{\arxiv}[1]{\href{http://arxiv.org/abs/#1}{\tt arXiv:\nolinkurl{#1}}}
\newcommand{\doi}[1]{\href{http://dx.doi.org/#1}{{\tt DOI:#1}}}
\newcommand{\euclid}[1]{\href{http://projecteuclid.org/getRecord?id=#1}{{\tt #1}}}
\newcommand{\mathscinet}[1]{\href{http://www.ams.org/mathscinet-getitem?mr=#1}{\tt #1}}
\newcommand{\googlebooks}[1]{(preview at \href{http://books.google.com/books?id=#1}{google books})}
\renewcommand{\googlebooks}[1]{}
\newcommand{\numdam}[1]{\href{http://www.numdam.org/item?id=#1}{\tt #1}}

% Theorems
\theoremstyle{plain}
\newtheorem{theorem}{Theorem}
\newtheorem{proposition}{Proposition}
\numberwithin{proposition}{section}
\newtheorem{lemma}[proposition]{Lemma}
\newtheorem{corollary}[proposition]{Corollary}
\newtheorem{claim}[proposition]{Claim}
\newtheorem{conjecture}[proposition]{Conjecture}
\newtheorem{observation}[proposition]{Observation}

\theoremstyle{definition}
\newtheorem{definition}[proposition]{Definition}
\newtheorem{exercise}[proposition]{Exercise}
\newtheorem{question}[proposition]{Question}
\newtheorem{problem}[proposition]{Problem}

\theoremstyle{remark}
\newtheorem{example}[proposition]{Example}
%\newtheorem{hint}[proposition]{Hint}
\newtheorem{remark}[proposition]{Remark}
%\newtheorem{apology}[proposition]{Apology}
%\newtheorem{warning}[proposition]{Warning}

% Hyphenation.
\hyphenation{Thurs-ton}

%ToDoNotes:
\newcommand{\nn}[1]{{\color{red}[[#1]]}}
\newcommand{\DPTtodo}[1]{\todo[color=green!40]{#1}}
\newcommand{\NStodo}[1]{\todo[color=blue!40]{#1}}
\newcommand{\SMtodo}[1]{\todo[color=red!40]{#1}}
\newcommand{\citationneeded}{\ \parbox{1.25in}{\todo[inline]{citation needed}}\ }
\newcommand{\referenceneeded}{\ \parbox{1.35in}{\todo[inline]{reference needed}}\ }

% Commands for exceptional paper
\newcommand{\Sk}[1]{\mathop{\mathrm{Sk}}(#1)}
\newcommand{\Skq}[1]{\mathop{\mathrm{Sk}_q}(#1)}
\newcommand{\Skcat}{\mathop{\mathsf{Sk}}}
\newcommand{\Skqcat}{\mathop{\mathsf{Sk}_q}}
\DeclareMathOperator{\eval}{eval}

\DeclareMathOperator{\Tw}{Tw}
\DeclareMathOperator{\HTw}{HTw}
\DeclareMathOperator{\Fr}{Fr}

\DeclareMathOperator{\fork}{fork}
\DeclareMathOperator{\fuse}{fuse}


%%% Local Variables: 
%%% mode: latex
%%% TeX-master: "main"
%%% TeX-master: t
%%% End: 

\DeclareMathOperator{\Tw}{Tw}
\DeclareMathOperator{\HTw}{HTw}
\DeclareMathOperator{\Fr}{Fr}

\begin{document}
\title{Multiplicities in the quantum exceptional series}

For the classical exceptional series, Deligne \cite{MR1378507} gives
decomposition formulas for up to the third tensor power of the adjoint
representation. He also gives information about plethysm (which
representation of the symmetric group things appear in). Stripping
the plethysm information out, we have

\begin{align*}
  X_1^{\otimes 0} &\simeq 1\\
  X_1^{\otimes 1} &\simeq X_1\\
  X_1^{\otimes 2} &\simeq 1 \oplus X_1 \oplus X_2 \oplus Y_2^{(*)}\\
  X_1^{\otimes 3} &\simeq 1 \oplus 5X_1 \oplus 4X_2 \oplus 3Y_2^{(*)} \oplus X_3
                    \oplus Y_3^{(*)} \oplus 3A \oplus 2C^{(*)}.
\end{align*}
The notation $Y_2^{(*)}$, for instance, means $Y_2 \oplus
Y_2^*$. Recall that $X_1 = \fg$.

In an exceptional quantum group, we expect $X_1^{\otimes n}$ to have
commuting representations of $U_q(\fg)$ and of $B_n$. Of course
representations of $B_n$ are much more complicated than
representations of $S_n$, so the simple plethysm-type formulas aren't
going to make a lot of sense. The four different copies of (for instance)
$X_2$ inside of $X_1^{\otimes 3}$, which appear in three different
representations of $S_3$ in the plethysm formulas, presumably come
from a single irreducible representation of $B_3$ which happens to
become reducible at $q=1$.

In our diagrammatic calculus, we have no direct access to the
representation $X_1^{\otimes 3}$, since every diagram that we write
down is by definition invariant. Instead, we can compute the 6-box
space, which is $\End(X_1^{\otimes 3})$ and comes with a left and
right action by $B_3$. All the multiplicities end up getting squared
when we do that. Note that
\[
80 = 1 + 5^2 + 4^2 + 2\cdot 3^2 + 1 + 2\cdot 1 + 3^2 + 2\cdot 2^2.
\]

To sort out which representation is which, we look at the center of
the braid group, the full-twist operator $\Tw_n$. In general, Braid
group representations break up as a direct sum of representations with
different eigenvalues of this full twist. It is also related to the
quadratic Casimir. To get that relation more precise, it's a little
better to look at a related operator, the framing change operator
$\Fr$, which does a Reidemeister 1 move to the entire bundle of
strands. This is the same as doing a full twist and adding a kink
(Reidemeister 1) to each strand individually. As such, we have
\[
\Fr = v^{-12n} \Tw,
\]
since the framing change on each strand contributes a factor of
$v^{-12}$.

For $n=2$, we also have the half-twist operator $\HTw_2$, which does
only a half-twist. (This makes sense in general, but is only central
for $n=2$.)

The data for all representations up to the third tensor power is shown
in Table~\ref{tab:e-vals}. The ``Casimir'' column comes directly from
Deligne (using $a = -\lambda/6$ and $a^* = (-1+\lambda)/6$). For the
representations appearing in $\fg^{\otimes 2}$, the eigenvalue for the
half-twist was computed directly. There is an evident relation between
the the Casimir eigenvalue and the framing change eigenvalue, which
should follow by considering the Kontsevich integral. The factor of
$-12$ presumably comes from the choice of how to normalize a
bubble. On the remaining lines, the same pattern relating the Casimir
to the framing change was extended.

\begin{table}
  \centering
\medskip
\begin{tabular}{cccccccccc}
  \toprule
      &         &   \multicolumn{4}{c}{Eigenvalues}    & \multicolumn{4}{c}{Multiplicities} \\
  \cmidrule(lr){3-6} \cmidrule(l){7-10}
  Rep & Casimir & $\Fr$ & $\HTw_2$ & $\Tw_2$ & $\Tw_3$ & $\fg^{\otimes0}$ & $\fg^{\otimes1}$ & $\fg^{\otimes2}$ & $\fg^{\otimes3}$\\ \midrule
  1   & 0       & 1     & $v^{12}$ & $v^{24}$ & $v^{36}$
      & 1 & 0 & 1 & 1\\[3pt]
  $\fg=X_1$ & 1 & $v^{-12}$ & $-v^6$ & $v^{12}$ & $v^{24}$
      &   & 1 & 1 & 5\\[3pt]
  $X_2$ & 2     & $v^{-24}$ & $-1$ & $1$      & $v^{12}$
      &   &   & 1 & 4\\[3pt]
  $Y_2$ & $2 - \frac{\lambda}{3}$ & $v^{-24}w^4$ & $w^2$ & $w^4$ & $v^{12}w^4$
      &   &   & 1 & 3\\[3pt]
  $Y_2^*$ & $\frac{5}{3} + \frac{\lambda}{3}$ & $v^{-20}w^{-4}$ & $v^2 w^{-2}$ & $v^4 w^{-4}$ & $v^{16}w^{-4}$
      &   &   & 1 & 3\\[3pt]
  $X_3$ & $3$ & $v^{-36}$ & & & $1$ &&&& 1\\[3pt]
  $Y_3$ & $3 - \lambda$ & $v^{-36}w^{12}$ &&& $w^{12}$ &&&& 1\\[3pt]
  $Y_3^*$ & $2 + \lambda$ & $v^{-24}w^{-12}$ &&& $v^{12}w^{-12}$ &&&& 1\\[3pt]
  $A$ & $\frac{8}{3}$ & $v^{-32}$ &&& $v^4$ &&&& 3\\[3pt]
  $C$ & $3 - \frac{\lambda}{2}$ & $v^{-36}w^6$ &&& $w^6$ &&&& 2\\[3pt]
  $C^*$ & $\frac{5}{2} + \frac{\lambda}{2}$ & $v^{-30}w^{-6}$ &&& $v^6w^{-6}$ &&&& 2\\[3pt]
  \bottomrule
\end{tabular}
\medskip
\caption{Eigenvalues and multiplicities for representations appearing in $\fg^{\otimes n}$ for $n \le 3$.}
\label{tab:e-vals}
\end{table}

\begin{table}
  \centering
  \begin{tabular}{ccccccccc}
    \toprule
    &&&\multicolumn{5}{c}{Braid gen e-val.} \\ \cmidrule(l){4-8}
    Rep & Mult & $\Tw_3$ & $v^{12}$ & $-v^{-6}$ & $-1$ & $w^2$ & $v^2w^{-2}$ \\
    \midrule
    $1$ & 1 & $v^{36}$ & 0 & 1 & 0 & 0 & 0 \\[2pt]
    $X_1$ & 5 & $v^{24}$ & 1 & 1 & 1 & 1 & 1 \\[2pt]
    $X_2$ & 4 & $v^{12}$ & 0 & 1 & 1 & 1 & 1 \\[2pt]
    $Y_2$ & 3 & $v^{12}w^4$ & 0 & 1 & 1 & 1 & 0 \\[2pt]
    $Y_2^*$ & 3 & $v^{16}w^{-4}$ & 0 & 1 & 1 & 0 & 1 \\[2pt]
    $X_3$ & 1 & $1$ & 0 & 0 & 1 & 0 & 0 \\[2pt]
    $Y_3$ & 1 & $w^{12}$ & 0 & 0 & 0 & 1 & 0 \\[2pt]
    $Y_3^*$ & 1 & $v^{12}w^{-12}$ & 0 & 0 & 0 & 0 & -1 \\[2pt]
    $A$ & 3 & $v^4$ & 0 & 0 & 1 & 1 & 1\\[2pt]
    $C$ & 2 & $w^6$ & 0 & 0 & 1 & 1 & 0\\[2pt]
    $C^*$ & 2 & $v^6w^{-6}$ & 0 & 0 & 1 & 0 & 1\\
    \bottomrule
  \end{tabular}
  \medskip
  \caption{Decomposition of the third tensor power into eigenspaces of a braid generator}\label{tab:braid-gen}
\end{table}

For each of the representations appearing in the third tensor power,
we can deduce the eigenspace decomposition of the braid
generator. Suppose we have a $d$-dimensional action of the braid
group, and consider the determinant of the full twist operator. If the
braid generator has eigenvalues $\lambda_1,\dots,\lambda_d$ and the
full twist has eigenvalue $\lambda_{\text{tot}}$, this is
\[
(\lambda_1\lambda_2\cdots\lambda_d)^6 = (\lambda_{\text{tot}})^d.
\]
On the other hand, each $\lambda_i$ must be in the set
$\{v^{12},-v^{-6},-1,w^2,v^2w^{-2}\}$, since we can factor through a
projection on to one of the irreducible representations appearing in
$\fg^{\otimes2}$. In each case, this uniquely determines the
$\lambda_i$. The results are shown in Table~\ref{tab:braid-gen}. We
could also read this off from Deligne's formulas for decomposition of
tensor powers. For instance, reading down the ``$-1$'' eigenvalue
column of Table~\ref{tab:braid-gen} tells us that
\[
X_2 \otimes \fg \simeq X_1 \oplus X_2 \oplus Y_2^{(*)} \oplus X_3
\oplus A \oplus C^{(*)},
\]
since $X_2$ is the representation in $\fg^{\otimes2}$ where the braid
generator has eigenvalue $-1$.

We can also conjecturally do the fourth tensor power. Cohen--de-Man
\cite{MR1381778} say that the fourth tensor power should decompose as
\begin{multline*}
\fg^{\otimes4} = 5\cdot 1 \oplus 16X_1 \oplus 23X_2 \oplus 18Y_2^{(*)}
  \oplus 21A \oplus 16C^{(*)} \oplus 12X_3 \oplus 6Y_3^{(*)} \\
  \oplus 6D^{(*)} \oplus 8E \oplus 6F^{(*)} \oplus 3G^{(*)} \oplus 2H^{(*)}
    \oplus 3I^{(*)} \oplus 3J \oplus X_4 \oplus Y_4^{(*)}
\end{multline*}

This is compatible (by computing dimension of endomorphism spaces)
with the conjectured dimensions of the $n$-box space, following
Cohen--de-Man's claim that exceptional representations of $F_4$ do not
degenerate up to the fourth tensor power:
\[
\begin{tabular}{rr}
  \toprule
  $n$ & Dim\\ \midrule
  0 & 1 \\ 1 & 0 \\ 2 & 1 \\ 3 & 1 \\ 4 & 5 \\ 5 & 16 \\
  6 & 80 \\ 7 & 436 \\ 8 & 2891 \\ 9 & 22248 \\ 10 & $\ge 198774$ \\
  \bottomrule
\end{tabular}
\]
For the representations newly appearing in the fourth tensor power, we
have Casimirs and eigenvalues below. The decomposition formulas also
tell us the eigenvalues of the full twist on three strands within each
of these representations and the eigenvalues of a braid
generator. It's not clear what that buys us, however.

\begin{table}
  \centering
  \medskip
  \begin{tabular}{ccccccccccc}
    \toprule
%     &&&&&\multicolumn{5}{c}{Twist eigenvals}\\ \cmidrule(l){6-10}
    Rep & Mult & Casimir & $\Fr$ & $\Tw_4$ \\ \midrule
%      $v^{12}$ & $-v^6$ & $1$ & $w^2$ & $v^2w^{-2}$ \\ \midrule
    $J$ & 3 & $\frac{20}{6}$ & $v^{-40}$ & $v^8$ \\[3pt]
    $E$ & 8 & $\frac{21}{6}$ & $v^{-42}$ & $v^6$ \\[3pt]
    $F$ & 6 & $\frac{22 - 2\lambda}{6}$ & $v^{-44}w^4$ & $v^4w^4$ \\[3pt]
    $D$ & 6 & $\frac{22 - 4\lambda}{6}$ & $v^{-44}w^8$ & $v^4w^8$ \\[3pt]
    $X_4$ & 1 & $\frac{24}{6}$ & $v^{-48}$ & 1\\[3pt]
    $I$ & 3 & $\frac{24 - 4\lambda}{6}$ & $v^{-48}w^8$ & $w^8$\\[3pt]
    $H$ & 2 & $\frac{24 - 6\lambda}{6}$ & $v^{-48}w^{12}$ & $w^{12}$\\[3pt]
    $G$ & 3 & $\frac{24 - 8\lambda}{6}$ & $v^{-48}w^{16}$ & $w^{16}$ \\[3pt]
    $Y_4$ & 1 & $\frac{24 - 12\lambda}{6}$ & $v^{-48}w^{24}$ & $w^{24}$ \\[3pt]
    \bottomrule
  \end{tabular}
  \medskip
  \caption{Representations in the fourth tensor power. The
    representations have been sorted by the value of the Casimir.}
  \label{tab:fourth-tensor}
\end{table}

\bibliographystyle{amsalpha}
\bibliography{bibliography/bibliography}

\end{document}


%%% Local Variables:
%%% mode: latex
%%% TeX-master: t
%%% End:
