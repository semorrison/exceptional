%auto-ignore
%this ensures the arxiv doesn't try to start TeXing here.
%!TEX root=trivalent.tex

\usepackage{amsmath,amssymb,amsfonts,amsthm}
\usepackage{ifpdf}
\usepackage{enumerate}
\usepackage{ulem}
\usepackage{leftidx}

\usepackage{breqn}

\usepackage{tikz}
\usetikzlibrary{calc}
\usepackage{tikz-qtree}


\ifpdf
\usepackage[pdftex,plainpages=false,hypertexnames=false,pdfpagelabels]{hyperref}
\else
\usepackage[dvips,plainpages=false,hypertexnames=false]{hyperref}
\fi


% ----------------------------------------------------------------
\vfuzz2pt % Don't report over-full v-boxes if over-edge is small
\hfuzz2pt % Don't report over-full h-boxes if over-edge is small
% ----------------------------------------------------------------

% diagrams -------------------------------------------------------
% figures ---------------------------------------------------------
%%% borrowed from Dror's cobordisms paper, use this to include eps or pdf graphics.
\newcommand{\pathtodiagrams}{diagrams/pdf/}

\newcommand{\mathfig}[2]{{\hspace{-3pt}\begin{array}{c}%
  \raisebox{-2.5pt}{\includegraphics[width=#1\textwidth]{\pathtodiagrams #2}}%
\end{array}\hspace{-3pt}}}

\newcommand{\inputtikz}[1]{\input{diagrams/tikz/#1.tex}}

\newcommand{\arxiv}[1]{\href{http://arxiv.org/abs/#1}{\tt arXiv:\nolinkurl{#1}}}
\newcommand{\doi}[1]{\href{http://dx.doi.org/#1}{{\tt DOI:#1}}}
\newcommand{\euclid}[1]{\href{http://projecteuclid.org/getRecord?id=#1}{{\tt #1}}}
\newcommand{\mathscinet}[1]{\href{http://www.ams.org/mathscinet-getitem?mr=#1}{\tt #1}}
\newcommand{\googlebooks}[1]{(preview at \href{http://books.google.com/books?id=#1}{google books})}

% THEOREMS -------------------------------------------------------
\theoremstyle{plain}
%\newtheorem*{fact}{Fact}
\newtheorem{prop}{Proposition}[section]
\newtheorem{conj}[prop]{Conjecture}
\newtheorem{thm}[prop]{Theorem}
\newtheorem{recipe}[prop]{Recipe}
\newtheorem{lem}[prop]{Lemma}
\newtheorem{alg}[prop]{Algorithm}
\newtheorem{cor}[prop]{Corollary}
\newtheorem*{cor*}{Corollary}
%\newtheorem*{example}{Example}
\newtheorem{question}{Question}
\newtheorem*{kuperberg}{The Kuperberg Program}
\newenvironment{rem}{\\ \noindent\textsl{Remark.}}{}  % perhaps looks better than rem above?
\numberwithin{equation}{section}
%\numberwithin{figure}{section}

\theoremstyle{remark}
\newtheorem{ex}[prop]{Example}
\newtheorem*{exc}{Exercise}
\newtheorem{remark}[prop]{Remark}           
\newtheorem*{rem*}{Remark}               %unnumbered remark
\newtheorem*{ex*}{Example}                %unnumbered exercise

\theoremstyle{definition}
\newtheorem{defn}[prop]{Definition}         % numbered definition
\newtheorem{assumption}[prop]{Assumption}   
\newtheorem{nota}[prop]{Notation}   
\newtheorem*{defn*}{Definition}             % unnumbered definition


\theoremstyle{plain}
\newcommand{\noqed}{\renewcommand{\qedsymbol}{}}

% Marginal notes in draft mode -----------------------------------
\newcommand{\scott}[1]{\stepcounter{comment}{{\color{green} $\star^{(\arabic{comment})}$}}\marginpar{\color{green}  $\star^{(\arabic{comment})}$ \usefont{T1}{scott}{m}{n}  #1 --S}}     % draft mode
\newcommand{\eep}[1]{\stepcounter{comment}{\color{blue} $\star^{(\arabic{comment})}$}\marginpar{\color{blue}  $\star^{(\arabic{comment})}$  #1 --E}}     % draft mode
\newcommand{\comment}[1]{\stepcounter{comment}$\star^{(\arabic{comment})}$\marginpar{\tiny $\star^{(\arabic{comment})}$ #1}}     % draft mode
\newcounter{comment}
\newcommand{\noop}[1]{}
\newcommand{\todo}[1]{\textcolor{blue}{\textbf{TODO: #1}}}
\newcommand{\nn}[1]{\textcolor{red}{[[#1]]}}

% \mathrlap -- a horizontal \smash--------------------------------
% For comparison, the existing overlap macros:
% \def\llap#1{\hbox to 0pt{\hss#1}}
% \def\rlap#1{\hbox to 0pt{#1\hss}}
\def\clap#1{\hbox to 0pt{\hss#1\hss}}
\def\mathllap{\mathpalette\mathllapinternal}
\def\mathrlap{\mathpalette\mathrlapinternal}
\def\mathclap{\mathpalette\mathclapinternal}
\def\mathllapinternal#1#2{%
\llap{$\mathsurround=0pt#1{#2}$}}
\def\mathrlapinternal#1#2{%
\rlap{$\mathsurround=0pt#1{#2}$}}
\def\mathclapinternal#1#2{%
\clap{$\mathsurround=0pt#1{#2}$}}

% MATH -----------------------------------------------------------
\newcommand{\Natural}{\mathbb N}
\newcommand{\Integer}{\mathbb Z}
\newcommand{\Rational}{\mathbb Q}
\newcommand{\Real}{\mathbb R}
\newcommand{\Complex}{\mathbb C}
\newcommand{\Field}{\mathbb F}

% tricky way to iterate macros over a list
\def\semicolon{;}
\def\applytolist#1{
    \expandafter\def\csname multi#1\endcsname##1{
        \def\multiack{##1}\ifx\multiack\semicolon
            \def\next{\relax}
        \else
            \csname #1\endcsname{##1}
            \def\next{\csname multi#1\endcsname}
        \fi
        \next}
    \csname multi#1\endcsname}

% \def\cA{{\cal A}} for A..Z
\def\calc#1{\expandafter\def\csname c#1\endcsname{{\mathcal #1}}}
\applytolist{calc}QWERTYUIOPLKJHGFDSAZXCVBNM;
% \def\bbA{{\mathbb A}} for A..Z
\def\bbc#1{\expandafter\def\csname bb#1\endcsname{{\mathbb #1}}}
\applytolist{bbc}QWERTYUIOPLKJHGFDSAZXCVBNM;
% \def\bfA{{\mathbf A}} for A..Z
\def\bfc#1{\expandafter\def\csname bf#1\endcsname{{\mathbf #1}}}
\applytolist{bfc}QWERTYUIOPLKJHGFDSAZXCVBNM;


\newcommand{\id}{\boldsymbol{1}}
\renewcommand{\imath}{\mathfrak{i}}
\renewcommand{\jmath}{\mathfrak{j}}

\newcommand{\qRing}{\Integer[q,q^{-1}]}
\newcommand{\qMod}{\qRing-\operatorname{Mod}}
\newcommand{\ZMod}{\Integer-\operatorname{Mod}}

\newcommand{\into}{\hookrightarrow}
\newcommand{\onto}{\twoheadrightarrow}
\newcommand{\iso}{\cong}
\newcommand{\actsOn}{\circlearrowright}

\newcommand{\htpy}{\simeq}

\newcommand{\abs}[1]{\left|#1\right|}
\newcommand{\norm}[1]{\left|\left|#1\right|\right|}
\newcommand{\ip}[1]{\left< #1\right>}

\newcommand{\restrict}[2]{#1{}_{\mid #2}{}}
\newcommand{\set}[1]{\left\{#1\right\}}
\newcommand{\setc}[2]{\left\{#1 \;\left| \; #2 \right. \right\}}
\newcommand{\quotient}[2]{\frac{#1}{#2}}
\newcommand{\relations}[2]{\left<#1 \;\left| \; #2 \right. \right>}
\newcommand{\cone}[3]{C\left(#1 \xrightarrow{#2} #3\right)}
\newcommand{\pairing}[2]{\left\langle#1 ,#2 \right\rangle}

\newcommand{\code}[1]{{\tt #1}}
\newcommand{\MMA}{\code{Mathematica} {}}

\newcommand{\bigraph}[1]{{\hspace{-3pt}\begin{array}{c}%
  \raisebox{-2.5pt}{\includegraphics[height=6mm]{\pathtographs \hashlookup{#1}}}% 
\end{array}\hspace{-3pt}}}

\newcommand{\graph}[1]{{\hspace{-3pt}\begin{array}{c}%
  \raisebox{-2.5pt}{\includegraphics[height=10mm]{\pathtographs \hashlookup{#1}}}% 
\end{array}\hspace{-3pt}}}

\newcommand{\J}[2]{\mathcal{J}_{#1,#2}}
\newcommand{\Jf}[2]{\mathcal{J}^{\text{framed}}_{#1,#2}}
\newcommand{\HOMFLY}{\operatorname{HOMFLYPT}}
\newcommand{\Kauffman}{\operatorname{Kauffman}}
\newcommand{\Dubrovnik}{\operatorname{Dubrovnik}}
\newcommand{\Rep}{\operatorname{Rep}}
\newcommand{\vRep}{\Rep^{\text{vector}}}

\newcommand{\overcrossing}{\inputtikz{overcrossing}}
\newcommand{\undercrossing}{\inputtikz{undercrossing}}
\newcommand{\identity}{\inputtikz{identity}}
\newcommand{\cupcap}{\inputtikz{cupcap}}
\newcommand{\Oovercrossing}{\inputtikz{Oovercrossing}}
\newcommand{\Oundercrossing}{\inputtikz{Oundercrossing}}
\newcommand{\Oidentity}{\inputtikz{Oidentity}}

\newcommand{\positivetwist}{\inputtikz{positivetwist}}
\newcommand{\negativetwist}{\inputtikz{negativetwist}}
\newcommand{\verticalstrand}{\inputtikz{verticalstrand}}

\newcommand{\card}[1]{\sharp{#1}}

\newcommand{\bdy}{\partial}
\newcommand{\compose}{\circ}
\newcommand{\eset}{\emptyset}
\newcommand{\disj}{\sqcup}

\newcommand{\psmallmatrix}[1]{\left(\begin{smallmatrix} #1 \end{smallmatrix}\right)}

\newcommand{\qiq}[2]{[#1]_{#2}}
\newcommand{\qi}[1]{\qiq{#1}{q}}
\newcommand{\qdim}{\operatorname{dim_q}}

\newcommand{\directSum}{\oplus}
\newcommand{\DirectSum}{\bigoplus}
\newcommand{\tensor}{\otimes}
\newcommand{\Tensor}{\bigotimes}

\newcommand{\altTensor}{\widehat{\Tensor}}

\newcommand{\db}[1]{\left(\left(#1\right)\right)}

\newcommand{\su}[1]{\mathfrak{su}_{#1}}
\newcommand{\csl}[1]{\mathfrak{sl}_{#1}}
\newcommand{\uqsl}[1]{U_q\left(\csl{#1}\right)}

\newcommand{\Cobl}{{\mathcal Cob}_{/l}}
\newcommand{\Cob}[1]{{\mathcal Cob}\left(\su{#1}\right)}
\newcommand{\Kom}[1]{\operatorname{Kom}\left(#1\right)}

\newcommand{\Mat}[1]{\operatorname{\mathbf{Mat}}\left(#1\right)}
\newcommand{\Inv}[1]{\operatorname{Inv}\left(#1\right)}
\newcommand{\Hom}{\operatorname{Hom}}
\newcommand{\End}[1]{\operatorname{End}\left(#1\right)}
\newcommand{\Reigenvalue}[2]{R_{#1 \subset #2 \tensor #2}}
\newcommand{\tildeReigenvalue}[2]{\tilde{R}_{#1 \subset #2 \tensor #2}}
\newcommand{\im}{\operatorname{im}}

\newcommand{\lk}[2]{\operatorname{lk}\left(#1,#2\right)}
\newcommand{\fr}[1]{\operatorname{fr}\left(#1\right)}

\newcommand{\asbimod}[2]{\operatorname{Mod}'\left(#1 \subset #2\right)}
\newcommand{\sbimod}[2]{\operatorname{Mod}\left(#1 \subset #2\right)}
\newcommand{\abimod}[2]{#1-\operatorname{Mod}'-#2}
\newcommand{\bimod}[2]{#1-\operatorname{Mod}-#2}
\newcommand{\bimodule}[3]{\leftidx{_#1}{#2}{_#3}}

\newcommand{\pa}{\mathcal{PA}}
\newcommand{\TL}{\mathcal{TL}}
\newcommand{\JW}[1]{f^{(#1)}}
\newcommand{\tr}[1]{\text{tr}(#1)}
\newcommand{\dn}[1]{{\mathcal D}{\mathit (#1)}}

\newcommand{\directSumStack}[2]{{\begin{matrix}#1 \\ \DirectSum \\#2\end{matrix}}}
\newcommand{\directSumStackThree}[3]{{\begin{matrix}#1 \\ \DirectSum \\#2 \\ \DirectSum \\#3\end{matrix}}}

\newcommand{\grading}[1]{{\color{blue}\{#1\}}}
\newcommand{\shift}[1]{\left[#1\right]}

\newcommand{\tensorover}[1]{\otimes_{#1}}
\newcommand{\tensorhat}{\widehat{\Tensor}}

\newcommand{\LL}{\mathcal{L}}
\newcommand{\Lhat}{\hat{\mathcal{L}}}
\newcommand{\writhe}{\operatorname{writhe}}

\newenvironment{narrow}[2]{%
\vspace{-0.4cm}% horrible hack, by scott % this only seems to be appropriate in beamer mode...
\begin{list}{}{%
\setlength{\topsep}{0pt}%
\setlength{\leftmargin}{#1}%
\setlength{\rightmargin}{#2}%
\setlength{\listparindent}{\parindent}%
\setlength{\itemindent}{\parindent}%
\setlength{\parsep}{\parskip}}%
\item[]}{\end{list}}


%%% futzing with margins following Dror (from Karoubi)
%\marginparwidth 0pt%
%\marginparsep 0pt

\textwidth   5.5in%
\textheight  9.0in%
\oddsidemargin 12pt%
\evensidemargin 12pt

\topmargin -.6in%
\headsep .5in
% ----------------------------------------------------------------