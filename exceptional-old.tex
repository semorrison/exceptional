
\section{Old outline proposal}

Old stuff here, banished from the main file

\vspace{1cm}
\hrule
\vspace{1cm}

Almost current proposed outline:
\begin{enumerate}
\item Introduction
\begin{enumerate}
\item Two-variable knot polynomials. For a two-variable knot polynomial to
exist, we first need a family of Lie algebras, and second need a coherent
quantisation of the family. We propose that this can be achieved in the 
exceptional series, and give many calculations of the conjectured knot
invariant.
\item Deligne's classical exceptional series, generators, relations, and 
conjectures.
\item The main result of this paper is a conjecture.
The quantum exceptional series, generators and relations and conjectures.
\item The rest of the paper consists of two kinds of evidence for this 
conjecture.
\begin{enumerate}
\item First, we show that the proposed description of the quantum exceptional 
series is the right one --- it is picked out as the unique braided tangle
category
with ...
\item Second, we give some weak evidence that the conjecture is viable.
\begin{enumerate}
\item We propose bases for the $n$-box spaces, $n \leq 6$ (or is it 7?), 
showing that these are linear independent, and their spans are closed under
various operations.
\item We show that these give representations of the affine $n$-strand braid 
groups, for $n \leq 6$.
\item We calculate ridiculously many two-variable knot polynomials; although 
these are conjectural, they unconditionally specialise to the correct quantum
knot polynomials for the exceptional Lie algebras.
\end{enumerate}
\item Mention the Kontsevich integral argument?
\end{enumerate}
\end{enumerate}
\item Main results
\begin{enumerate}
\item The classification statement.
\item We defer some corner cases to the end.
\item Proof of the main case.
\item Statement of the evaluation and consistency conjectures.
\item Some consequences: other relations.
\item Specializations.
  \begin{enumerate}
  \item $U_q(\fg)$ for $\fg$ in Deligne's list in the adjoint
    representation
  \item The 7-dimensional representation of $U_q(\fg_2)$
  \item $U_q(\fg)$ for $\fg$ in the $F_4$ family
    At $q=1$, these have $v^6=-1$ (since the trivalent vertex is
    symmetric). This family has only a finite number of points.
  \end{enumerate}
\end{enumerate}
\item Calculations
\begin{enumerate}
\item Heuristics for choosing bases for $n$-box spaces
\item Verifying linear independence, and invariance under operations
\item A representation of the affine 6-strand braid group
\item 2-variable polynomials for knots with Conway width at most 6.
\begin{enumerate}
\item Define Conway graph and Conway width.
\item Explain that we can calculate anything with Conway width at most 6.
\item This includes all 12-crossing knots, and all but one 12-crossing link.
\item Mention Westbury's paper, and the other paper that has done some knot
polynomial calculations.
\end{enumerate}
\item Denominators.
\item Dimensions of irreducible objects.
\end{enumerate}
\item A Kontsevich integral argument: the classical conjecture implies the
quantum conjecture.
\item Corner cases from the classification.
\end{enumerate}


\vspace{1cm}
\hrule
\vspace{1cm}


Dylan's earlier proposal for an outline.
\begin{enumerate}
\item Introduction
  \begin{enumerate}
  \item Quantum exceptional results and conjectures
    \begin{enumerate}
    \item Skein relations
      \begin{enumerate}
      \item Framing change
      \item New relation (state uniqueness)
      \item Implies both classical relations in $q \to 1$ limit
      \item Mention other consequences, including double-crossing switch
      \end{enumerate}
    \item Conjectures: Reduction, consistency, $n$-box space
    \item Classical consistency $\Rightarrow$ quantum consistency.

      Kontsevich integral argument
    \end{enumerate}
  \item Dimensions of $n$-box spaces
  \item Evaluations we can compute easily
    \begin{enumerate}
    \item Exceptional computations hard previously
    \item With double-clasp switch
    \item Knots with width $\le $ 3
    \item Knots and links with Conway graph width $\le 6$,
      including all 11-crossing diagrams and all 12-crossing
      knots.
    \end{enumerate}
  \end{enumerate}
\item Classical exceptional series
  \begin{enumerate}
  \item Jacobi and Vogel relations
  \item Reduction conjecture
  \item Consistency conjecture
  \item Something about the $n$-box space, or perhaps delay. (Could
    ask for finite-dimensionality, or positive-definite pairing)
  \end{enumerate}
\item Quantum exceptional relation: Uniqueness given dimensions of
  spaces
\item Computations for $n$-box space, $n \le 6$
  \begin{enumerate}
  \item Inner products approach
  \item Explicit derivation approach
  \item Denominators
  \end{enumerate}
\item Kontsevich integral argument: classical conjecture implies
  quantum conjecture
\item Computations for knots with Conway width $\le 6$
  \begin{enumerate}
  \item Define Conway graph of a diagram, Conway width
  \item Show that 11-crossing knots have Conway width $\le 6$
  \end{enumerate}
\item Known points on the quantum exceptional family
  \begin{enumerate}
  \item $U_q(\fg)$ for $\fg$ in Deligne's list in the adjoint
    representation
  \item $U_q(\fg)$ for $\fg$ in the $F_4$ family

    At $q=1$, these have $v^6=-1$ (since the trivalent vertex is
    symmetric). This family has only a finite number of points.
  \end{enumerate}
\end{enumerate}

\section{Introduction}
\label{sec:introduction}

In this paper, we study a conjectural \emph{quantum exceptional
  series}. To give a first statement of the conjecture, consider the
field of rational functions in two variables $v$ and~$w$, and define
\begin{align*}
[k\lambda + n] &= w^kv^n - w^{-k}v^{-n}\\
\{k\lambda + n\} &= w^k v^n + w^{-k} v^{-n} = \frac{[2k+2n]}{[k+n]}.
\end{align*}
This is analogous to ``quantum integer'' notation, adapted to a
two-variable setting, but note that there is no denominator.
\DPTtodo{Will need names for field of rational functions, as well as
  rings with various denominators}

Now consider the vector space of framed, embedded trivalent graphs in
$\RR^3$, modulo local skein relations that let us reduce unknots,
monogons, bigons, and triangles, and allow us to change framings along
edges and at vertices:
  \begin{equation}
    \label{eq:simple-rels-spec}
  \begin{aligned}
    \unknot\; &= d&\quad
    \loopvertex\;&=0\\[5pt]
      \ngon{2}\;&= b\;\onestrandid&
        \ngon[-90]{3}\; &= t\;\nvertex[-90]{3}\\[5pt]
      \twist\; &= v^{12}\;\onestrandid&
        \twistvertex\; &= -v^{6}\;\nvertex[-90]{3}
  \end{aligned}
  \end{equation}
where
\begin{align*}
  d &= -\frac{\{2\}[\lambda+5][\lambda-6]}{[\lambda][\lambda-1]}\\
  b &= \frac{\{\lambda+2\}\{\lambda-3\}[3]}{[1]}\\
  t &= \{1\}\bigl(\{1\}w^2/v + (v^4 - v^2 - 1 - v^{-2} + v^{-4}) +
      \{1\}v/w^2\bigr).
\end{align*}
To these skein relations, we add the \emph{quantum IHX relation}
\begin{equation}
v^{-3} \;
\drawcrossX
\;+ v \;
\drawI
\; -v^{-1} \;
 \drawH
\;
 - \frac{[\lambda][\lambda-1]}{[1]}
\left[\; \braidcross \;
 + v^{4}\;
\cupcap
\; + v^{-4} \;
 \twostrandid \;
 \right] = 0.\label{eq:quant-except-spec}
\end{equation}

We conjecture that the skein relations above suffice to evaluate
uniquely every framed knotted trivalent graph. In fact we split the
conjecture up into different parts.

\DPTtodo{Introduced some placeholder terminology here so I could state
  the conjectures more precisely. Feel free to change it.}
Let $R$ be the ring $\QQ[v,w,v - v^{-1},w - w^{-1}, w/v - v/w]$, that
is, the minimal ring for which all of the denominators in the skein
relations are defined. Let $\Skq{n,m}$ be the \emph{skein algebra vector space}
of framed, embedded trivalent $(n,m)$ tangles modulo the
local skein relations in \eqref{eq:simple-rels-spec}
and~\eqref{eq:quant-except-spec}. The different vector spaces
$\Skq{n,m}$ combine into a category $\Skqcat$ where the objects are
integers and the morphisms from $n$ to $m$ are elements of $\Skq{n,m}$.

\begin{conjecture}[Quantum consistency]
  \label{conj:quant-consist}
  There is a category $\mathsf{QExcept}$ where the objects are
  integers and morphisms are finite-dimensional, free $R$-modules, and
  a functor $\eval$ from $\Skqcat$ to $M_n$, that is the identity on
  objects and surjective on morphisms after extending scalars to
  $\QQ(v,w)$. Furthermore, $\mathsf{QExcept}(n,m)$ has dimension
  $1,\allowbreak0,\allowbreak1,\allowbreak1,\allowbreak5,\allowbreak16,\allowbreak80$
  for $n+m=0,1,2,3,4,5,6$, respectively, and $\eval$ takes a single
  strand and a trivalent vertex to generators of the respective
  morphism spaces.
\end{conjecture}

\DPTtodo{Is this too weak? To derive square reduction, we do assume
  something more about the denominators, so we probably do need to
  extend scalars at least a little. Would be nice to have a version of
  sufficiency that doesn't assume consistency}
\begin{conjecture}[Quantum sufficiency]
  \label{conj:quant-suffic}
  After extending scalars to $\QQ(v,w)$, the functor $\eval$
  conjectured above is injective on each morphism space. In
  particular, these relations suffice to evaluate any closed framed
  trivalent graph to a rational function in $v$ and~$w$.
\end{conjecture}

  The relations above are quite general. With some mild assumptions on
  the base ring, any skein theory where the $n$-box space has
  dimension at most $1,0,1,1,5$ for $n=0,1,2,3,4$ satisfies a version
  of relation~\eqref{eq:quant-except-spec}. See
  Theorem~\ref{thm:quant-except} in Section~\ref{sec:relation} for a
  precise statement.

\section{The classical exceptional series}
\label{sec:classical-except}

(Introduce Jacobi skein from Lie algebras.)

For any Lie algebra, the skein theory satisfies the \emph{Jacobi} (or \emph{IHX})
relation:
\begin{equation}
\drawI\; - \;\drawH\; + \;\drawcrossX\; = 0.
\label{eq:IHX}
\end{equation}
We will scale the value of a vertex so that a bubble has the
value~$6$, which implies the value of a trivalent bubble.
\begin{align}
\ngon{2}\; &= 6\;\onestrandid &
  \ngon{3}\; &= 3\;\nvertex{3}
  \label{eq:classical-bigon-trigon}
\end{align}

In addition, Vogel observed that all the exceptional Lie algebras
satisfy another relation, the \emph{classical exceptional relation},
which can be put in the form
\begin{equation}
\ngon[45]{4}\; = \;\drawI\; + \;\drawH\;
 + \omega \left[ \;\cupcap\; + \;\twostrandid\; + \;\symcross\; \right]
\label{eq:classical-except-old}
\end{equation}
for varying values of~$w$.

\begin{theorem}[Vogel]
  For the following Lie groups, the skein theory of the Lie bracket in
  the adjoint representation
  satisfies the classical exceptional relation with the given values
  of $\mu$ and~$w$.
  \[
  \begin{tabular}{lcc}
    \toprule
    Group         & $\mu$ & $\omega$\\
    \midrule
    Trivial             & $5$ & $15$ \\
    $O(1 \mid 2)$       & $4$ & $10$ \\
    $\PSL(2)$           & $3$ & $6$ \\
    $\PSL(3)\rtimes\ZZ/2$& $2$ & $3$ \\
    $G_2$              & $3/2$ & $15/8$ \\
    $PO(8)\rtimes\ZZ/3$  & $1$ & $1$\\
    $F_4$               & $2/3$ & $5/9$\\
    $E_6\rtimes\ZZ/2$   & $1/2$ & $3/8$\\
    $E_7$               & $1/3$ & $2/9$ \\
    $E_8$               & $1/5$ & $3/25$ \\
    \bottomrule
  \end{tabular}
  \]
\end{theorem}
In each case, the relevant group has the minimal representation
theory: representations in the root lattice (i.e., appearing in
decompositions of the adjoint representation), and invariant under
outer automorphisms (or symmetries of the Dynkin diagram).

\begin{remark}
  Almost every paper on the exceptional series introduces at least one new
  parameter for the series. We hold with this tradition by introducing the
  parameter~$w$ above. For reference, here is a list of the various
  parameters that have been used, along with how they are related to
  the parameter $\mu$ and who introduced them. It is often natural to
  parameterize the exceptional series with a parameter that is
  ambiguous, with two values for each group; if this applies, the
  involution relating the two values is also listed.
  \begin{enumerate}
  \item $a_D = \mu/6$; involution $a_D \leftrightarrow -a_D-1/6$ \cite{MR1378507}.
  \item $\lambda = -\mu$; involution $\lambda \leftrightarrow 1-\lambda$ \cite{MR1378507}.
  \item $\mu$ as in the table above; involution $\mu \leftrightarrow -1-\mu$ \cite{MR1411045}.
  \item $\nu = 1/\mu$; involution $\nu \leftrightarrow -\nu/(\nu+1)$
    \cite{MR1952563}.
  \item The dual Coxeter number $h^\vee = 6/\mu$
    \cite{MR1952563}.
  \item The number $\omega$ in Equation~\eqref{eq:classical-except-old}, with
    $\omega=\mu(1+\mu)/2$. This appears to be the first time it has been
    published, but it appeared in earlier (unavailable?) preprint
    versions of Vogel's paper \cite{MR2769234}.
  \item\label{item:eigenvalues} The two non-trivial eigenvalues $\alpha$ and $\beta$ of the ladder
    operator $\psi_L$ on $\Sym^2\fg$, defined by
    \[
    \psi_L\left(\;\idtangle\;\right) = \;\ladder\;.
    \]
    With vertices normalized by
    Equation~\eqref{eq:classical-bigon-trigon}, we have $\alpha = -\mu$ and
    $\beta = 1+\mu$ \cite{MR2769234}.
    The involution switches $\alpha$ and $\beta$. The usual quadratic
    Casimir is $12 - 2\psi_L$.
  \item The dimension of the Lie algebra $d = \dim \mathfrak{g} =
    -2\frac{(\mu-5)(\mu+6)}{\mu(\mu+1)}$.
  \item The dimension of an algebra related to triality, $a_{LM} =
    2(1-\mu)/\mu$ \cite{MR1933384}.
  \item $m = 6(1+\mu)/\mu$; involution $m/6 \leftrightarrow 6/m$
    \cite[Chapter 17]{MR2418111}.
  \end{enumerate}
  The essential issue is that with only a finite number of points to
  work with, there is no natural accumulation point to put at infinity
  and base the family around.
\end{remark}

\begin{remark}
  Vogel took item \ref{item:eigenvalues} on the list above as his
  starting point. It is easy to deduce
  Equation~\eqref{eq:classical-except-old} (with $\omega = -\alpha\beta/2$)
  from the condition on eigenvalues of $\psi_L$.
\end{remark}

\nn{
Introduce skein theory over $\QQ(\omega)$ (or $\QQ[\omega]$?). Conjecture that
it is complete (everything is equivalent to a polynomial) and
consistent (no polynomials are consequences).
}

\DPTtodo{Figure out the Conjecture in \cite[Section
  7.5]{MR2769234}. Does it relate?}

\section{The quantum exceptional relation}
\label{sec:relation}

The main equation we are interested in is the \emph{quantum
  exceptional relation}:
\begin{equation}
v^{-3} \;
\drawcrossX
\;+ v \;
\drawI
\; -v^{-1} \;
 \drawH
\;
 + \alpha
\left[\; \braidcross \;
 + v^{4}\;
\cupcap
\; + v^{-4} \;
 \twostrandid \;
 \right] = 0.\label{eq:quant-except}
\end{equation}
In fact, this relation turns out to be quite universal.

For the remainder of this section, suppose that we are working in a
skein theory on embedded, framed trivalent graphs.

\begin{proposition}
  If the dimension of the $n$-box space for $n=0,1,2,3$ is $1,0,1,1$,
  respectively, then we have the relations
  \begin{equation}
    \label{eq:simple-rels}
  \begin{aligned}
    \unknot\; &= d&\qquad
      \twist\; &= s^2\;\onestrandid&\qquad
        \twistvertex\; &= -s\;\nvertex[-90]{3}\\[5pt]
    \loopvertex\;&=0&
      \ngon{2}\;&= b\;\onestrandid&
        \ngon[-90]{3}\; &= t\;\nvertex[-90]{3}
  \end{aligned}
  \end{equation}
  for some parameters $d, b, t, s$ in the base ring.
\end{proposition}

\begin{proof}
\NStodo{Add claim: trivalent vertex is rotationally symmetric}
  Most of these are immediate from the assumption. The one fact that
  is not immediate is the relation between the coefficients of
  twisting a vertex ($-s$ above) and changing framing ($s^2$
  above). Twisting a vertex twice can be turned into three framing
  changes, one on each of the adjacent strands.
\DPTtodo{This is standard, but needs pictures or a reference}
\end{proof}

\begin{theorem}\label{thm:quant-except}
  Suppose that $Q$ is a skein theory over a field on embedded, framed
  trivalent
  graphs so that the dimension of the $n$-box space for $n=0,1,2,3,4$
  is at most $1,0,1,1,5$, respectively. Suppose that the scalar $s$
  above has three distinct cube roots and at least one sixth root.
  Then either $Q$ vanishes on all trivalent
  graphs or it satisfies the quantum exceptional relation
  \eqref{eq:quant-except} for some choice of~$v$ with $v^6 = s$.
\end{theorem}

In fact, the conditions we need for the conclusion of
Theorem~\ref{thm:quant-except} are somewhat weaker than stated. In
particular, we
don't need to know the exact dimensions of the $n$-box spaces, just
that certain diagrams
are linearly dependent. We do not spell out the details here.

\begin{proof}
  Let $v$ be a sixth root of $s$ (arbitrary for the moment).
  By assumption, the space spanned by the six diagrams
  \[
  \cupcap\;,\qquad\twostrandid\;,\qquad\braidcross\;,
    \qquad\drawI\;,\qquad\drawH\;,\qquad\drawcrossX\;
  \]
  is at most $5$-dimensional.  These six diagrams should be thought
  of as having the symmetries of a tetrahedron (up to framing
  change). More precisely, consider the operation~$R$ that cyclically
  rotates three of the input strands.
  \[
  R\left(\,\,\idtangle\,\,\right) = \Rcycle.
  \]
  The operation~$R$ permutes the six diagrams above up to powers of $\pm
  v^k$:
  \begin{align*}
    \begin{tikzpicture}
      \node[inner sep=5pt] (A) at (150:1.4cm) {\cupcap};
      \node[inner sep=5pt] (B) at (30:1.4cm) {\twostrandid};
      \node[inner sep=5pt] (C) at (-90:1.4cm) {\braidcross};
      \draw[|->,bend left=15] (A) to node[above,cdlabel] {v^{-12}} (B);
      \draw[|->,bend left=15] (B) to (C);
      \draw[|->,bend left=15] (C) to (A);
    \end{tikzpicture}&&
    \begin{tikzpicture}
      \node[inner sep=5pt] (D) at (150:1.4cm) {\drawI};
      \node[inner sep=5pt] (E) at (30:1.4cm) {\drawH};
      \node[inner sep=5pt] (F) at (-90:1.4cm) {\drawcrossX};
      \draw[|->,bend left=15] (D) to node[above,cdlabel] {-v^{-6}} (E);
      \draw[|->,bend left=15] (E) to node[below right,cdlabel] {\!\!-v^{-6}} (F);
      \draw[|->,bend left=15] (F) to (D);
    \end{tikzpicture}
  \end{align*}
  The notation means, for instance, that
  \[
  R\left(\,\,\cupcap\,\,\right) = v^{-12}\,\,\twostrandid\,\,.
  \]
  Notice that $R^3$ acts by multiplication by $v^{-12}$. (Indeed,
  $R^3$ does not permute the strands, and is equivalent by
  Reidemeister moves to changing the
  framing on the upper-right strand by~$-1$.)

  In particular, the possible
  eigenvalues for $R$ are $v^{-4}$, $\omega_3 v^{-4}$, and
  $\omega_3^{-1} v^{-4}$, where $\omega_3$ is a primitive cube root of
  unity (which exists by assumption on the base ring). At least one of
  the three eigenspaces is at
  most one-dimensional. By possibly changing the choice of $v$ as a
  root of $-v^6$, we can assume that the eigenspace for $v^{-4}$ is at
  most one-dimensional. This then implies that there is a relation of
  the form
  \begin{equation*}
\beta \left[v^{-3} \;
\drawcrossX
\;+ v \;
\drawI
\; -v^{-1} \;
 \drawH
\;\right]
 + \alpha
\left[\; \braidcross \;
 + v^{4}\;
\cupcap
\; + v^{-4} \;
 \twostrandid \;\right]=0.
  \end{equation*}

  If $\beta \ne 0$, we are done. If $\beta = 0$, the relation reduces
  to the Kauffman bracket relation for the Jones polynomial:
\[
\braidcross \;
 + v^{4}\;
\cupcap
\; + v^{-4} \;
 \twostrandid\;=0.
\]
Closing this relation off with a trivalent vertex on the bottom yields
\begin{align*}
  \twistvertex + v^{-4}\; \nvertex[-90]{3} &= 0,
\end{align*}
so $-v^6 + v^{-4} = 0$, or $v^{10}=1$,
assuming that a trivalent vertex is not~$0$.
\NStodo{Explain that you get the golden category, and still have
  quantum exceptional relation in a degenerate way}
\end{proof}

\begin{proposition}
  If a skein theory satisfies the quantum exceptional
  relation~\eqref{eq:quant-except} and the reduction
  relations~\eqref{eq:simple-rels} with $v^{40} \ne 1$, then the
  parameters satisfy
  \begin{align}
    s &= v^6\label{eq:s-v-rel}\\
    b &= -\frac{\alpha(d + v^8 + v^{-8})}{v^5 - v^{-5}}\label{eq:b-rel}\\
%    b(v^5 - v^{-5}) + \alpha d + \alpha (v^8 + v^{-8}) &= 0\label{eq:b-rel}\\
    t &= \frac{b - \alpha(v^5 - v^{-5})}{v^2 + v^{-2}}\label{eq:b-t-rel}.
%    b - t(v^2 + v^{-2}) - \alpha (v^5 - v^{-5}) &= 0.\label{eq:b-t-rel}
  \end{align}
\end{proposition}

\begin{proof}
  Equation~\eqref{eq:s-v-rel} is part of
  Theorem~\ref{thm:quant-except}. Equation~\eqref{eq:b-rel} follows
  from closing off Equation~\eqref{eq:quant-except} with a cap, and
  then solving for~$b$. Equation~\eqref{eq:b-t-rel} follows from
  closing off Equation~\eqref{eq:quant-except} by attaching two ends
  to a trivalent vertex, and then solving for~$t$.

  The condition that $v^{40} \ne 1$ guarantees that we do not divide
  by~$0$.
\end{proof}


\section{Scratch}
Here are some things remaining for Dylan (or maybe Noah) to write
up/think about:
\begin{itemize}
\item Deriving the square-reduction relation using the quantum exceptional
\item Deriving the crossing change relation from the quantum
  exceptional relation, or maybe from another dimensionality (or
  maybe eigenvalue) argument
\item The F4--E6 family in this context
\item Showing there is an eigenvalue for the twist with eigenvalue
  $-1$, and finding the eigenvector
\item Naturally introducing a new parameter, $w$, an eigenvalue of
  rotation, and using it here to get nice formulas. Use ability to
  renormalize vertices.
\item Do a write-up that the classical existence conjecture implies
  the quantum existence conjecture, using the Kontsevich integral.
\end{itemize}

Here are some things Scott is hoping to work on soon:
\begin{itemize}
\item write a link evaluator using the clasp switch relation (in progress)
\item compute the action of the 3-strand braid group on diagrams with 4
boundary points
\begin{itemize}
\item find the eigenvalues of the generators and the scalar by which the full twist
acts; compare against Tuba-Wenzl \cite{MR1815266}.
\end{itemize}
\item try to compute the 3-strand braid group action on a basis for the 3-boxes
\begin{itemize}
\item perhaps even try to compute the structure coefficients for a basis for
the 3-boxes?
\end{itemize}
\item find the value of dodecahedron
\begin{itemize}
\item by computing the determinant of inner
products of 80 elements, including the difference of the two threepents, to
obtain a linear identity in the polyhedron
\item but linear algebra in
rational functions in 2 variables is very slow
\item perhaps finding the value at particular points and interpolating is worth
a try?
\item perhaps linear algebra with rational functions in \emph{one} variable,
then interpolating for the second, is also worth a try!
\end{itemize}
\item write a human readable summary of the computer calculation that braided +
$1,0,1,1,5,\leq16$ implies a planar pentasquare reduction formula
\item derive the planar pentasquare reduction formula directly from
  quantum IHX

  See the note \nolinkurl{2013-07-27-reducing-5-gon.pdf}.
\begin{quote}
``From your formula to get a planar pentasquare reduction, first note the second term can be rewritten using Jacobi as a difference of two planar diagrams.  Then we need to look at the third term, which is harder (there has to be one hard term because we haven't used Vogel's relation yet).  Use Vogel's relation to replace the square with a crossing and planar stuff.  This gives five terms,  three of which are easily rewritten to be planar.  Of the remaining two, one of them you can pull a Y through a crossing and then use Vogel to rewrite the crossing as planar stuff, and the other you can use Vogel directly on the crossing to get planar terms including a pentasquare.''
---Noah
\end{quote}
\end{itemize}
Feel free to advise or criticize these plans!


\begin{conjecture}[Quantum consistency]
  \label{conj:quant-consist}
  There is a category $\mathsf{QExcept}$ where the objects are
  integers and morphisms are finite-dimensional, free $R$-modules, and
  a functor $\eval$ from $\Skqcat$ to $M_n$, that is the identity on
  objects and surjective on morphisms after extending scalars to
  $\QQ(v,w)$. Furthermore, $\mathsf{QExcept}(n,m)$ has dimension
  $1,\allowbreak0,\allowbreak1,\allowbreak1,\allowbreak5,\allowbreak16,\allowbreak80$
  for $n+m=0,1,2,3,4,5,6$, respectively, and $\eval$ takes a single
  strand and a trivalent vertex to generators of the respective
  morphism spaces.
\end{conjecture}

\DPTtodo{Is this too weak? To derive square reduction, we do assume
  something more about the denominators, so we probably do need to
  extend scalars at least a little. Would be nice to have a version of
  sufficiency that doesn't assume consistency}
\begin{conjecture}[Quantum sufficiency]
  \label{conj:quant-suffic}
  After extending scalars to $\QQ(v,w)$, the functor $\eval$
  conjectured above is injective on each morphism space. In
  particular, these relations suffice to evaluate any closed framed
  trivalent graph to a rational function in $v$ and~$w$.
\end{conjecture}


%Did you and your collaborators provide preprints that have evolved during, or from, the work done during your stays at HIM?
%If so, then please inform Christian Wegner (wegner@him.uni-bonn.de), who will add your article to the preprint list on the HIM website.
  

\begin{proposition}
Suppose $\cC$ satisfies QEJacobi for some $v$ and $\alpha$ with $v$ not a $10$th or $12$th root of unity, then the following equations hold.
\begin{itemize}
  \item
The quantum exceptional square relation:
\begin{equation*} \ngon[45]{4} + \frac{\alpha \widetilde{\Phi}_{12} (b+[3]\alpha)}{\widetilde{\Phi}_1 \widetilde{\Phi}_3} \widetilde{\Phi}_8 \braidcross + \frac{y_1}{\widetilde{\Phi}_3} \widetilde{\Phi}_8 \drawI + \frac{y_2}{\widetilde{\Phi}_3} \widetilde{\Phi}_8 \drawH + \frac{y_3}{\widetilde{\Phi}_1 \widetilde{\Phi}_3} \widetilde{\Phi}_8 \cupcap + \frac{y_4}{\widetilde{\Phi}_1 \widetilde{\Phi}_3} \widetilde{\Phi}_8 \twostrandid =0
\end{equation*}
where the $y_i$ are Laurent polynomials in $v$ and $\alpha$ given in Appendix \ref{app:coefficients}.
  \item
The quantum exceptional crossing relation:
\begin{equation*}
\braidcross - \invbraidcross + \frac{\widetilde{\Phi}_1 \widetilde{\Phi}_3 \widetilde{\Phi}_4 \widetilde{\Phi}_8}{b+[3]\alpha} \left[\; \drawI \; - \; \drawH \; \right] - \frac{\widetilde{\Phi}_1^2 \widetilde{\Phi}_3 \widetilde{\Phi}_4 \widetilde{\Phi}_8 \alpha}{b+[3]\alpha} \left[\; \twostrandid \; - \; \cupcap \; \right] = 0
\end{equation*}
\end{itemize}
 \end{proposition}
\begin{proof}
First attach an H-diagram to the QEJacobi relation in the three natural ways respecting the symmetries of the tetrahedron, and then we take a cleverly chosen linear combination of them so that the twisted squares cancel.

\begin{align}
\label{eq:deriving-square-relation}
\mathfig{0.8}{deriving-square-relation-1} \\
= \mathfig{0.8}{deriving-square-relation-2}
\end{align}

We now use the QEJacobi relation again to remove the X-term, and then divide by $[3]$ (which is nonzero because $v$ is not a $12$th root of unity), yielding the QESquare relation.

Take the QESquare relation and subtract its 90-degree rotation, yielding:

\begin{equation*}
\frac{\widetilde{\Phi}_{12} \alpha(b+[3]\alpha)}{\{2\}} \left(\braidcross - \invbraidcross\right) + [6]\alpha \left[\; \drawI \; - \; \drawH \; \right] - [1][6] \alpha^2 \left[\; \twostrandid \; - \; \cupcap \; \right] = 0
\end{equation*}

Since $v$ is not a $12$th root of unity, $\widetilde{\Phi}_{12} \neq 0$, since $v$ is not a $10$th root of unity $\alpha$ is non-zero.  


Assume $b+ [3] \alpha = 0$ for the sake of contradiction.  Then the above equation divided by the non-zero number $[6] \alpha$ becomes

\begin{equation*}
\left[\; \drawI \; - \; \drawH \; \right] - [1]\alpha \left[\; \twostrandid \; - \; \cupcap \; \right] = 0
\end{equation*}

This means we're in the setting of the previous lemma.  Since $v$ is not a $12$th or $10$th root of unity, the only remaining cases are all quotients $SO(3)_q$ with the standard braiding (this includes the Golden category).  Now a direct calculation shows that $b+[3] \alpha = b \frac{\widetilde{\Phi}_12}{v^4-1+v^{-4}}$ which is non-zero since is not a $12$th root of unity.

%Capping this off, we see that $b -[1] \alpha (d-1) = 0$.  So, $\frac{\alpha}{b} = \frac{1}{[1](d-1)}$ but also $\frac{\alpha}{b} = -\frac{[5]}{d+\{8\}}$ and $\frac{\alpha}{b} = - \frac{1}{[3]}$.   So, $d = -\{2\}$ and $\frac{\alpha}{b} = -\frac{1}{[3]}$
\end{proof}
