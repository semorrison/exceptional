\documentclass[12pt]{amsart}
\usepackage{amssymb}
\usepackage{cite}
\usepackage{array}
\usepackage{booktabs}
\usepackage{mdwtab}
\usepackage{mathtools}
\usepackage[T1]{fontenc}
\usepackage[utf8]{inputenc}
\usepackage{hyphenat}
\usepackage{enumitem}
\usepackage{ifpdf}
\ifpdf
  \usepackage[pdftex]{graphicx}
  \usepackage[pdftex,margin=1in]{geometry}
  \usepackage[bookmarks=true, bookmarksopen=true,%
    bookmarksdepth=3,bookmarksopenlevel=2,%
    colorlinks=true,%
    linkcolor=blue,%
    citecolor=blue,%
    filecolor=blue,%
    menucolor=blue,%
    urlcolor=blue]{hyperref}
  \hypersetup{pdftitle={The quantum exceptional series}}
  \hypersetup{pdfauthor={Scott Morrison, Noah Snyder, and Dylan P. Thurston}}
\else
  \usepackage[dvips]{graphicx}
  \usepackage[dvips,margin=1in]{geometry}
  % Use hyperref with all features turned off even in DVI mode, since
  % the .aux file format changes
  \usepackage[draft]{hyperref}
\fi
\usepackage{url}

% A binary operator with a subscript on both sides (and correct spacing)
% Name stands for subscript-operator-subscript
\newcommand{\sos}[3]{\mathbin{{}_{#1}\mathord#2_{#3}}}

% manyindices
% Adapted from code by "bza" in comp.text.tex, Feb. 7, 2006
%% USAGE:
%%
%% \manyindices#1#2#3#4#5
%%
%% #1=lower left index
%% #2=upper left index
%% #3=lower right index
%% #4=upper right index
%% #5=main symbol
\makeatletter
\newcommand\mi@kern[1]{%
  \settowidth\@tempdima{$\mi@obj^{#1}$}
  \kern-\@tempdima
  #1
  \settowidth\@tempdima{$\mi@obj$}
  \kern\@tempdima
}

\newtoks\mi@toksp
\newtoks\mi@toksb
\DeclareRobustCommand{\manyindices}[5]{
  \def\mi@obj{#5}
  \mi@toksp\expandafter{\mi@kern{#2}}
  \mi@toksb\expandafter{\mi@kern{#1}}
  \@mathmeasure4\textstyle{#5_{#1}^{#2}}
  \@mathmeasure6\textstyle{#5_{#3}^{#4}}
  \dimen0-\wd6 \advance\dimen0\wd4
  \@mathmeasure8\textstyle{\hphantom{{}_{#1}^{#2}}#5^{\the\mi@toksp#4}_{\the\mi@toksb#3}}
  \hbox to \dimen0{}{\kern-\dimen0\box8}
}
\makeatother 

% Left sub/super scripts
% \lsup is a temporary definition until something better is worked out
% Use \lsupv if the next argument is vertical
\newcommand{\lsub}[2]{{}_{#1}#2}
\newcommand{\lsup}[2]{{}^{#1}\mskip-.6\thinmuskip#2}
\newcommand{\lsupv}[2]{{}^{#1}#2}
\newcommand{\lsubsup}[3]{\manyindices{#1}{\mskip.6\thinmuskip#2\mskip-.6\thinmuskip}{}{}{\mathord{#3}}}
\newcommand{\lsubsupv}[3]{\manyindices{#1}{\mskip.2\thinmuskip#2\mskip-.2\thinmuskip}{}{}{\mathord{#3}}}

\newcounter{saveenum}

% Read the file, if it exists
\newread\testin
\def\maybeinput#1{
\openin\testin=#1
\ifeof\testin\typeout{Warning: input #1 not found}\else\input#1\fi
\closein\testin
}

\def\mathcenter#1{%
  \vcenter{\hbox{$#1$}}%
}

\def\graph#1{
        \includegraphics{#1}
}

\def\mathgraph#1{
        \mathcenter{\graph{#1}}
}

\def\mfig#1{
        \mathcenter{\includegraphics{#1}}
}

\def\mfigb#1{
        \mathcenter{\includegraphics[trim=-1 -1 -1 -1]{#1}}
}


%%% Local Variables: 
%%% mode: latex
%%% TeX-master: "main"
%%% End: 

% General use
\newcommand{\RR}{\mathbb R}
\newcommand{\CC}{\mathbb C}
\newcommand{\ZZ}{\mathbb Z}
\newcommand{\QQ}{\mathbb Q}
\newcommand{\PP}{\mathbb P}
\newcommand{\EE}{\mathbb E}
\newcommand{\HH}{\mathbb H}
\newcommand{\NN}{\mathbb N}

\newcommand{\comma}{\mathbin ,}
\newcommand{\conn}{\mathbin \#}
\newcommand{\sltwo}{{{\mathfrak{sl}}_2}}
\renewcommand{\sl}{\mathfrak{sl}}
\newcommand{\gl}{\mathfrak{gl}}
\newcommand{\fg}{{\mathfrak g}}
\newcommand{\co}{\colon\thinspace}
\newcommand{\eps}{\varepsilon}
\newcommand{\abs}[1]{{\lvert #1 \rvert}}
\newcommand{\norm}[1]{{\lVert #1 \rVert}}
\newcommand{\OneHalf}{{\textstyle\frac{1}{2}}}

% Synonyms for commands I never remember
\newcommand{\isom}{\cong}
\newcommand{\superset}{\supset}
\newcommand{\bigcircle}{\bigcirc}
\newcommand{\contains}{\ni}
\newcommand{\tensor}{\otimes}
\newcommand{\bdy}{\partial}

% Stupid overloading.
\newcommand{\lbracket}{[}
\newcommand{\rbracket}{]}

% Various operators.
\DeclareMathOperator{\ad}{ad}
\DeclareMathOperator{\Ad}{Ad}
\DeclareMathOperator{\End}{End}
\DeclareMathOperator{\sign}{sign}
\DeclareMathOperator{\Sym}{Sym}
\DeclareMathOperator{\tr}{tr}
\DeclareMathOperator{\Hom}{Hom}
\DeclareMathOperator{\vol}{vol}
\DeclareMathOperator{\rank}{rank}
\DeclareMathOperator{\im}{im}

% Linear groups
\DeclareMathOperator{\ISO}{\mathit{ISO}}
\DeclareMathOperator{\SO}{\mathit{SO}}
\DeclareMathOperator{\GL}{\mathit{GL}}
\DeclareMathOperator{\SL}{\mathit{SL}}
\DeclareMathOperator{\PSL}{\mathit{PSL}}

% citations
\newcommand{\arxiv}[1]{\href{http://arxiv.org/abs/#1}{\tt arXiv:\nolinkurl{#1}}}
\newcommand{\doi}[1]{\href{http://dx.doi.org/#1}{{\tt DOI:#1}}}
\newcommand{\euclid}[1]{\href{http://projecteuclid.org/getRecord?id=#1}{{\tt #1}}}
\newcommand{\mathscinet}[1]{\href{http://www.ams.org/mathscinet-getitem?mr=#1}{\tt #1}}
\newcommand{\googlebooks}[1]{(preview at \href{http://books.google.com/books?id=#1}{google books})}
\renewcommand{\googlebooks}[1]{}
\newcommand{\numdam}[1]{\href{http://www.numdam.org/item?id=#1}{\tt #1}}

% Theorems
\theoremstyle{plain}
\newtheorem{theorem}{Theorem}
\newtheorem{proposition}{Proposition}
\numberwithin{proposition}{section}
\newtheorem{lemma}[proposition]{Lemma}
\newtheorem{corollary}[proposition]{Corollary}
\newtheorem{claim}[proposition]{Claim}
\newtheorem{conjecture}[proposition]{Conjecture}
\newtheorem{observation}[proposition]{Observation}

\theoremstyle{definition}
\newtheorem{definition}[proposition]{Definition}
\newtheorem{exercise}[proposition]{Exercise}
\newtheorem{question}[proposition]{Question}
\newtheorem{problem}[proposition]{Problem}

\theoremstyle{remark}
\newtheorem{example}[proposition]{Example}
%\newtheorem{hint}[proposition]{Hint}
\newtheorem{remark}[proposition]{Remark}
%\newtheorem{apology}[proposition]{Apology}
%\newtheorem{warning}[proposition]{Warning}

% Hyphenation.
\hyphenation{Thurs-ton}

%ToDoNotes:
\newcommand{\nn}[1]{{\color{red}[[#1]]}}
\newcommand{\DPTtodo}[1]{\todo[color=green!40]{#1}}
\newcommand{\NStodo}[1]{\todo[color=blue!40]{#1}}
\newcommand{\SMtodo}[1]{\todo[color=red!40]{#1}}
\newcommand{\citationneeded}{\ \parbox{1.25in}{\todo[inline]{citation needed}}\ }
\newcommand{\referenceneeded}{\ \parbox{1.35in}{\todo[inline]{reference needed}}\ }

% Commands for exceptional paper
\newcommand{\Sk}[1]{\mathop{\mathrm{Sk}}(#1)}
\newcommand{\Skq}[1]{\mathop{\mathrm{Sk}_q}(#1)}
\newcommand{\Skcat}{\mathop{\mathsf{Sk}}}
\newcommand{\Skqcat}{\mathop{\mathsf{Sk}_q}}
\DeclareMathOperator{\eval}{eval}

\DeclareMathOperator{\Tw}{Tw}
\DeclareMathOperator{\HTw}{HTw}
\DeclareMathOperator{\Fr}{Fr}

\DeclareMathOperator{\fork}{fork}
\DeclareMathOperator{\fuse}{fuse}


%%% Local Variables: 
%%% mode: latex
%%% TeX-master: "main"
%%% TeX-master: t
%%% End: 


\begin{document}
\title{The quantum exceptional series}

\author[Morrison]{Scott Morrison}
\address{Mathematical Sciences Institute, Australian National University}
\email{scott.morrison@anu.edu.au}

\author[Snyder]{Noah Snyder}
\address{Bloomington, Indiana, USA}
\email{nsnyder1@indiana.edu} % Noah, do you prefer a different e-mail address?

\author[Thurston]{Dylan~P.~Thurston}
\address{Bloomington, Indiana, USA}
\email{dpthurst@indiana.edu}

\begin{abstract}
  We find a single two-parameter skein relation on trivalent graphs,
  the \emph{quantum IHX relation}, that specializes to a skein
  relation associated to each exceptional Lie algebra. If a slight
  strengthening of Deligne's conjecture on the existence of a
  (classical) exceptional series is true, then this relation
  determines a new two-variable quantum exceptional polynomial, at
  least as a power series near $q=1$. The
  single quantum IHX relation specializes to both the classical IHX
  (or Jacobi) relation that holds for every Lie algebra and to the
  Vogel relation holding for the exceptional series.

  We find a conjectural basis for the space of diagrams with $n$ loose
  ends modulo the quantum IHX relation for $0 \le n \le 7$, with
  dimensions agreeing with the classical computations. We use the
  skein relation to compute the (conjectural) quantum exceptional
  polynomial for all knots with up to $8$ crossings, in particular
  determining (unconditionally) the values of the quantum polynomials
  for the exceptional Lie algebras in the adjoint representation on
  these knots.
\end{abstract}

%\subjclass[2000]{Primary xxx; Secondary xxx}
%\keywords{}

\maketitle

\tableofcontents

\section{Scratch}
Here are some things Scott is hoping to work on soon:
\begin{itemize}
\item write a link evaluator using the clasp switch relation (in progress)
\item compute the action of the 3-strand braid group on diagrams with 4
boundary points
\item try to compute the 3-strand braid group action on a basis for the 3-boxes
\begin{itemize}
\item perhaps even try to compute the structure coefficients for a basis for
the 3-boxes?
\end{itemize}
\item find the value of dodecahedron
\begin{itemize}
\item by computing the determinant of inner
products of 80 elements, including the difference of the two threepents, to
obtain a linear identity in the polyhedron
\item but linear algebra in
rational functions in 2 variables is very slow
\item perhaps finding the value at particular points and interpolating is worth
a try?
\end{itemize}
\item write a human readable summary of the computer calculation that braided +
$1,0,1,1,5,\leq16$ implies a planar pentasquare reduction formula
\item derive the planar pentasquare reduction formula directly from quantum IHX
\end{itemize}
Feel free to advise or criticize these plans!

\bibliographystyle{hamsplain}
\bibliography{bibliography/bibliography}

\end{document}
