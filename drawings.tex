\usepackage{fp}
\usepackage{tikz}
\usetikzlibrary{matrix}
\usetikzlibrary{arrows,backgrounds,patterns,scopes,external,hobby,
    decorations.pathreplacing,
    decorations.pathmorphing
}

\newlength{\fuzzwidth}
\setlength{\fuzzwidth}{2.5pt}
\newlength{\arrowlength}
\setlength{\arrowlength}{8pt}
\newlength{\arrowwidth}
\setlength{\arrowwidth}{.75pt}
\newlength{\pointrad}
\setlength{\pointrad}{1.5pt}
\newlength{\linewid}
\setlength{\linewid}{1.5pt}
\newlength{\circlerad}
\setlength{\circlerad}{16pt}
\newlength{\smcirclerad}
\setlength{\smcirclerad}{8pt}
\newcommand{\fillcolor}{black!60}
\newcommand{\fuzzcolor}{black!25}
\newcommand{\arrowcolor}{black!25}
\newcommand{\covercolor}{black!0}
\newcommand{\graycolor}{black!55}
\newcommand{\graylightcolor}{black!40}


\newcommand{\coverwidthfuzz}{6pt}
\newcommand{\coverwidth}{3.5pt}
\newcommand{\coverwidththin}{3.25pt}
\newcommand{\coverwidththick}{3.75pt}

\newlength{\linewidthin}
\setlength{\linewidthin}{1.25pt}
\newlength{\linewidthick}
\setlength{\linewidthick}{1.75pt}


\tikzset{use Hobby shortcut}
\tikzset{
	coverline/.style={
	preaction={draw,line width=\coverwidth,\covercolor}}, 
	coverlinethin/.style={
	preaction={draw,line width=\coverwidththin,\covercolor}}, 
	coverlinethick/.style={
	preaction={draw,line width=\coverwidththick,\covercolor}}, 
	coverlineleft/.style={
	preaction={draw,line width=\coverwidthfuzz,\covercolor,decorate,decoration={curveto,amplitude=0,raise=.35*\fuzzwidth}}}, %%% Notice this number was tweaked, 	
coverlinelefttail/.style={
	preaction={draw,line width=\coverwidthfuzz,\covercolor,decorate,decoration={curveto,amplitude=0,raise=.35*\fuzzwidth,pre=moveto,pre length=2pt}}}, %%% Notice this number was tweaked, probably will not look good if parameters change.  Again here, pre removes the raise option.
        fuzzlefttail/.style={
        preaction={draw,line width=\fuzzwidth,\fuzzcolor,decorate,decoration={curveto,pre=moveto,pre length=2pt,amplitude=0,raise=.5*\fuzzwidth}}}, %%% This does not work --- somehow pre breaks the raise function.
        linestylethin/.style={line width=\linewidthin},
        linestylethick/.style={line width=\linewidthick},
        linestylegray/.style={line width=\linewid,\graycolor},
        linestylegraylight/.style={line width=\linewid,\graylightcolor}
}
\tikzset{
        fuzzright/.style={
        preaction={draw,line width=\fuzzwidth,\fuzzcolor,decorate,decoration={curveto,amplitude=0,raise=-.5*\fuzzwidth}}},
        fuzzleft/.style={
        preaction={draw,line width=\fuzzwidth,\fuzzcolor,decorate,decoration={curveto,amplitude=0,raise=.5*\fuzzwidth}}},
        fuzzrightpre/.style={ %%% Doesn't work
        preaction={draw,line width=2pt,\fuzzcolor,decorate,decoration={curveto,amplitude=0,raise=-1pt,pre=moveto,pre length=12pt}}},
        fuzzleftpre/.style={ %%% Doesn't work
        preaction={draw,line width=2pt,\fuzzcolor,decorate,decoration={curveto,post=moveto,post length=32pt,amplitude=0,raise=1pt}}},        
        outstyle/.style={\arrowcolor, line width=\arrowwidth},
        linestyle/.style={line width=\linewid}
}

\newcommand{\lilarrow}{
\draw[->] (0,0) -- (1,0);
}

\newcommand{\cb}{\raisebox{.6ex-.5\height}}


%%%% These draw triple or quadruple set of arrows of length 0.5 cm
\DeclareMathOperator{\rightdoublearrows} {{\; \begin{tikzpicture} \foreach \y in {0.05, 0.15} { \draw [-stealth] (0, \y) -- +(0.5, 0);} \; \end{tikzpicture}}}
\DeclareMathOperator{\leftdoublearrows} {{\; \begin{tikzpicture} \foreach \y in {0.05, 0.15} { \draw [stealth-] (0, \y) -- +(0.5, 0);} \; \end{tikzpicture}}}
\DeclareMathOperator{\righttriplearrows} {{\; \begin{tikzpicture} \foreach \y in {0, 0.1, 0.2} { \draw [-stealth] (0, \y) -- +(0.5, 0);} \; \end{tikzpicture}}}
\DeclareMathOperator{\lefttriplearrows} {{\; \begin{tikzpicture} \foreach \y in {0, 0.1, 0.2} { \draw [stealth-] (0, \y) -- +(0.5, 0);} \; \end{tikzpicture}}}
\DeclareMathOperator{\rightquadarrows} {{\; \begin{tikzpicture} \foreach \y in {-0.05, 0.05, 0.15, 0.25} { \draw [-stealth] (0, \y) -- +(0.5, 0);} \; \end{tikzpicture}}}
\DeclareMathOperator{\leftquadarrows} {{\; \begin{tikzpicture} \foreach \y in {-0.05, 0.05, 0.15, 0.25} { \draw [stealth-] (0, \y) -- +(0.5, 0);} \; \end{tikzpicture}}}

\newcommand{\ngon}[2][0]{
\begin{tikzpicture}[baseline=-0.5ex,scale=0.8]
\foreach \x in {1, ..., #2}
	\draw (360*\x/#2+#1:.8cm)--(360*\x/#2+#1:.5cm);
\foreach \x in {1, ..., #2}
	\draw (360*\x/#2+#1:.5cm) .. controls +(360*\x/#2+120+#1:.3cm) and +(360*\x/#2+360/#2-120+#1:.3cm) .. (360*\x/#2+360/#2+#1:.5cm);
\end{tikzpicture}
}

\newcommand{\nvertex}[2][0]{
\begin{tikzpicture}[baseline=-0.5ex,scale=0.8]
\foreach \x in {1, ..., #2}
	\draw (360*\x/#2+#1:.8cm)--(0,0);
\end{tikzpicture}
}

\newcommand{\unknot}{
\begin{tikzpicture}[baseline=-0.5ex,scale=0.8]
  \draw (0,0) circle (.6cm);
\end{tikzpicture}
}

\newcommand{\drawI}{ \begin{tikzpicture}[baseline=-0.5ex,scale=0.8]
 	\draw (0,.2) -- (45:.8cm);
 	\draw (0,.2) -- (135:.8cm);
	\draw (0,.2) -- (0,-.2);
 	\draw (0,-.2) -- (-45:.8cm);
 	\draw (0,-.2) -- (-135:.8cm);
\end{tikzpicture}
}

\newcommand{\drawH}{ \begin{tikzpicture}[baseline=-0.5ex,rotate=90,scale=0.8]
 	\draw (0,.2) -- (45:.8cm);
 	\draw (0,.2) -- (135:.8cm);
	\draw (0,.2) -- (0,-.2);
 	\draw (0,-.2) -- (-45:.8cm);
 	\draw (0,-.2) -- (-135:.8cm);
\end{tikzpicture}}

\newcommand{\onestrandid}{\begin{tikzpicture}[baseline=-0.5ex,scale=0.8]
	\draw (-.8cm,0)--(.8cm,0);
\end{tikzpicture}}

\newcommand{\twostrandid}{\begin{tikzpicture}[baseline=-0.5ex,scale=0.8]
	\draw (45:.8cm) to [curve through=(90:.3cm)] (135:.8cm);
	\draw (-45:.8cm) to [curve through=(-90:.3cm)] (-135:.8cm);
\end{tikzpicture}}

\newcommand{\cupcap}{\begin{tikzpicture}[baseline=-0.5ex,rotate=90,scale=0.8]
	\draw (45:.8cm) to [curve through=(90:.3cm)] (135:.8cm);
	\draw (-45:.8cm) to [curve through=(-90:.3cm)] (-135:.8cm);
\end{tikzpicture}}

\newcommand{\symcross}{\begin{tikzpicture}[baseline=-0.5ex,scale=0.8]
	\draw (45:.8cm) -- (-135:.8cm);
	\draw (-45:.8cm) -- (135:.8cm);
\end{tikzpicture}}

\newcommand{\braidcross}{\begin{tikzpicture}[baseline=-0.5ex,scale=0.8]
	\draw (45:.8cm) -- (-135:.8cm);
	\draw[line width=1mm,white,double=black] (-45:.8cm) -- (135:.8cm);
\end{tikzpicture}}


\newcommand{\drawcrossX}{ \begin{tikzpicture}[baseline=-0.5ex,scale=0.8,rotate=90]
 	\draw ([out angle=30].2,.3) .. (45:.8cm);
	\draw (.2,.3) -- (.2,-.3);
 	\draw ([out angle=-30].2,-.3) .. (-45:.8cm);
        \draw ([out angle=-150].2,-.3) ..([in angle=-70]135:.8cm);
 	\draw[line width=1mm,white,double=black]
              ([out angle=150].2,.3) .. ([in angle=70]-135:.8cm);
\end{tikzpicture}}

\newcommand{\twist}{
  \begin{tikzpicture}[baseline=-0.5ex,scale=0.8]
    \draw[line width=1mm,white,double=black]
       ([out angle=-15]135:.8cm)..([blank=soft]-60:.4cm)..(-120:.4cm)..([in angle=-165]45:.8cm);
    \draw[line width=1mm,white,double=black,use previous Hobby path={invert soft blanks,disjoint}];
  \end{tikzpicture}}

\newcommand{\twistvertex}{
\begin{tikzpicture}[baseline=-0.5ex,scale=0.8]
  \draw (0,-0.5)--(0,-0.8);
  \draw ([out angle=-170]30:0.8cm)..([in angle=150](0,-0.5);
  \draw[line width=1mm,white,double=black]
        ([out angle=-10]150:0.8cm)..([in angle=30](0,-0.5);
\end{tikzpicture}}

\newcommand{\loopvertex}{
\begin{tikzpicture}[baseline=-0.5ex,scale=0.8]
  \draw (0,-0.5)--(0,-0.8);
  \draw ([out angle=150]0,-0.5)..(-0.02,0.6)..(0.02,0.6)..([in angle=30]0,-0.5);
\end{tikzpicture}
}

\newcommand{\pentagon}{
\begin{tikzpicture}[scale=.12,baseline=-2]
\draw (36:1) -- (108:1) -- (180:1) -- (252:1) -- (-36:1) -- (36:1);
\end{tikzpicture}
}

\newcommand{\psq}{
\begin{tikzpicture}[scale=.15,baseline=-2]
\draw (36:1) -- (108:1) -- (180:1) -- (252:1) -- (-36:1) -- (36:1) -- +(1.2,0) -- ($(-36:1)+(1.2,0)$) -- (-36:1);
\end{tikzpicture}
}

\newcommand{\sqsq}{
\begin{tikzpicture}[scale=.15]
\draw (0,0) -- (2,0) -- (2,1) -- (0,1) -- cycle (1,0) -- (1,1);
\end{tikzpicture}
}
