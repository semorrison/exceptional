% !TEX root = exceptional.tex

\usepackage{fp}
\usepackage{tikz}
\usetikzlibrary{matrix}
\usetikzlibrary{arrows,backgrounds,patterns,scopes,external,hobby,
    decorations.pathreplacing,
    decorations.pathmorphing
}

\newlength{\fuzzwidth}
\setlength{\fuzzwidth}{2.5pt}
\newlength{\arrowlength}
\setlength{\arrowlength}{8pt}
\newlength{\arrowwidth}
\setlength{\arrowwidth}{.75pt}
\newlength{\pointrad}
\setlength{\pointrad}{1.5pt}
\newlength{\linewid}
\setlength{\linewid}{1.5pt}
\newlength{\circlerad}
\setlength{\circlerad}{16pt}
\newlength{\smcirclerad}
\setlength{\smcirclerad}{8pt}
\newcommand{\fillcolor}{black!60}
\newcommand{\fuzzcolor}{black!25}
\newcommand{\arrowcolor}{black!25}
\newcommand{\covercolor}{black!0}
\newcommand{\graycolor}{black!55}
\newcommand{\graylightcolor}{black!40}


\newcommand{\coverwidthfuzz}{6pt}
\newcommand{\coverwidth}{3.5pt}
\newcommand{\coverwidththin}{3.25pt}
\newcommand{\coverwidththick}{3.75pt}

\newlength{\linewidthin}
\setlength{\linewidthin}{1.25pt}
\newlength{\linewidthick}
\setlength{\linewidthick}{1.75pt}

\tikzset{cdlabel/.style={execute at begin node=$\scriptstyle,execute at end node=$}}

\tikzset{use Hobby shortcut}
\tikzset{
	coverline/.style={
	preaction={draw,line width=\coverwidth,\covercolor}}, 
	coverlinethin/.style={
	preaction={draw,line width=\coverwidththin,\covercolor}}, 
	coverlinethick/.style={
	preaction={draw,line width=\coverwidththick,\covercolor}}, 
	coverlineleft/.style={
	preaction={draw,line width=\coverwidthfuzz,\covercolor,decorate,decoration={curveto,amplitude=0,raise=.35*\fuzzwidth}}}, %%% Notice this number was tweaked, 	
coverlinelefttail/.style={
	preaction={draw,line width=\coverwidthfuzz,\covercolor,decorate,decoration={curveto,amplitude=0,raise=.35*\fuzzwidth,pre=moveto,pre length=2pt}}}, %%% Notice this number was tweaked, probably will not look good if parameters change.  Again here, pre removes the raise option.
        fuzzlefttail/.style={
        preaction={draw,line width=\fuzzwidth,\fuzzcolor,decorate,decoration={curveto,pre=moveto,pre length=2pt,amplitude=0,raise=.5*\fuzzwidth}}}, %%% This does not work --- somehow pre breaks the raise function.
        linestylethin/.style={line width=\linewidthin},
        linestylethick/.style={line width=\linewidthick},
        linestylegray/.style={line width=\linewid,\graycolor},
        linestylegraylight/.style={line width=\linewid,\graylightcolor}
}
\tikzset{
        fuzzright/.style={
        preaction={draw,line width=\fuzzwidth,\fuzzcolor,decorate,decoration={curveto,amplitude=0,raise=-.5*\fuzzwidth}}},
        fuzzleft/.style={
        preaction={draw,line width=\fuzzwidth,\fuzzcolor,decorate,decoration={curveto,amplitude=0,raise=.5*\fuzzwidth}}},
        fuzzrightpre/.style={ %%% Doesn't work
        preaction={draw,line width=2pt,\fuzzcolor,decorate,decoration={curveto,amplitude=0,raise=-1pt,pre=moveto,pre length=12pt}}},
        fuzzleftpre/.style={ %%% Doesn't work
        preaction={draw,line width=2pt,\fuzzcolor,decorate,decoration={curveto,post=moveto,post length=32pt,amplitude=0,raise=1pt}}},        
        outstyle/.style={\arrowcolor, line width=\arrowwidth},
        linestyle/.style={line width=\linewid}
}

\newcommand{\lilarrow}{
\draw[->] (0,0) -- (1,0);
}

\newcommand{\cb}{\raisebox{.6ex-.5\height}}


%%%% These draw triple or quadruple set of arrows of length 0.5 cm
\DeclareMathOperator{\rightdoublearrows} {{\; \begin{tikzpicture} \foreach \y in {0.05, 0.15} { \draw [-stealth] (0, \y) -- +(0.5, 0);} \; \end{tikzpicture}}}
\DeclareMathOperator{\leftdoublearrows} {{\; \begin{tikzpicture} \foreach \y in {0.05, 0.15} { \draw [stealth-] (0, \y) -- +(0.5, 0);} \; \end{tikzpicture}}}
\DeclareMathOperator{\righttriplearrows} {{\; \begin{tikzpicture} \foreach \y in {0, 0.1, 0.2} { \draw [-stealth] (0, \y) -- +(0.5, 0);} \; \end{tikzpicture}}}
\DeclareMathOperator{\lefttriplearrows} {{\; \begin{tikzpicture} \foreach \y in {0, 0.1, 0.2} { \draw [stealth-] (0, \y) -- +(0.5, 0);} \; \end{tikzpicture}}}
\DeclareMathOperator{\rightquadarrows} {{\; \begin{tikzpicture} \foreach \y in {-0.05, 0.05, 0.15, 0.25} { \draw [-stealth] (0, \y) -- +(0.5, 0);} \; \end{tikzpicture}}}
\DeclareMathOperator{\leftquadarrows} {{\; \begin{tikzpicture} \foreach \y in {-0.05, 0.05, 0.15, 0.25} { \draw [stealth-] (0, \y) -- +(0.5, 0);} \; \end{tikzpicture}}}

\newcommand{\newfig}[3]{
  \newsavebox{#2}
  \savebox{#2}{#3}
  \newcommand{#1}{\usebox{#2}}
}

\newfig{\twistedsquarehor}{\twistedsquarehorbox}{
\begin{tikzpicture}[baseline=-.5ex,scale=.8]
\draw (45:.8cm) -- (-135:.8cm);
\draw[line width=1mm,white,double=black] (-45:.8cm) -- (135:.8cm);
\draw (45:.5cm) .. controls +(-75:.3cm) and +(75:.3cm) .. (-45:.5cm);
\draw (135:.5cm) .. controls +(-105:.3cm) and +(105:.3cm) .. (-135:.5cm);
\end{tikzpicture}}

\newfig{\twistedsquarever}{\twistedsquareverbox}{
\begin{tikzpicture}[baseline=-.5ex,scale=.8]
\draw (45:.8cm) -- (-135:.8cm);
\draw[line width=1mm,white,double=black] (-45:.8cm) -- (135:.8cm);
\draw (45:.5cm) .. controls +(165:.3cm) and +(15:.3cm) .. (135:.5cm);
\draw (-45:.5cm) .. controls +(-165:.3cm) and +(-15:.3cm) .. (-135:.5cm);
\end{tikzpicture}}



\newcommand{\ngon}[2][0]{
\begin{tikzpicture}[baseline=-0.5ex,scale=0.8]
\foreach \x in {1, ..., #2}
	\draw (360*\x/#2+#1:.8cm)--(360*\x/#2+#1:.5cm);
\foreach \x in {1, ..., #2}
	\draw (360*\x/#2+#1:.5cm) .. controls +(360*\x/#2+120+#1:.3cm) and +(360*\x/#2+360/#2-120+#1:.3cm) .. (360*\x/#2+360/#2+#1:.5cm);
\end{tikzpicture}
}

\newcommand{\nvertex}[2][0]{
\begin{tikzpicture}[baseline=-0.5ex,scale=0.8]
\foreach \x in {1, ..., #2}
	\draw (360*\x/#2+#1:.8cm)--(0,0);
\end{tikzpicture}
}

\newfig{\twogon}{\twogonbox}{\ngon{2}}
\newfig{\threegon}{\threegonbox}{\ngon[-90]{3}}
\newfig{\threevertex}{\threevertexbox}{\nvertex[-90]{3}}
\newfig{\fourgon}{\fourgonbox}{\ngon[45]{4}}

\newfig{\unknot}{\unknotbox}{
\begin{tikzpicture}[baseline=-0.5ex,scale=0.8]
  \draw (0,0) circle (.6cm);
\end{tikzpicture}
}

\newfig{\drawI}{\drawIbox}{ \begin{tikzpicture}[baseline=-0.5ex,scale=0.8]
 	\draw (0,.2) -- (45:.8cm);
 	\draw (0,.2) -- (135:.8cm);
	\draw (0,.2) -- (0,-.2);
 	\draw (0,-.2) -- (-45:.8cm);
 	\draw (0,-.2) -- (-135:.8cm);
\end{tikzpicture}
}

\newfig{\drawH}{\drawHbox}{ \begin{tikzpicture}[baseline=-0.5ex,rotate=90,scale=0.8]
 	\draw (0,.2) -- (45:.8cm);
 	\draw (0,.2) -- (135:.8cm);
	\draw (0,.2) -- (0,-.2);
 	\draw (0,-.2) -- (-45:.8cm);
 	\draw (0,-.2) -- (-135:.8cm);
\end{tikzpicture}}

\newfig{\drawdottedH}{\drawdottedHbox}{ \begin{tikzpicture}[baseline=-0.5ex,scale=0.8]
        \draw (45:.8cm) -- (-45:.8cm);
        \draw (135:.8cm) -- (-135:.8cm);
        \draw[dotted] (-0.57cm,0) -- (0.57cm, 0);
\end{tikzpicture}}

\newfig{\onestrandid}{\onestrandidbox}{\begin{tikzpicture}[baseline=-0.5ex,scale=0.8]
	\draw (-.8cm,0)--(.8cm,0);
\end{tikzpicture}}

\newfig{\drawcup}{\drawcupbox}{\begin{tikzpicture}[baseline=-0.5ex,scale=0.8]
	\draw (45:.8cm) to [curve through=(90:0cm)] (135:.8cm);
\end{tikzpicture}}

\newfig{\cupcap}{\cupcapbox}{\begin{tikzpicture}[baseline=-0.5ex,scale=0.8]
	\draw (45:.8cm) to [curve through=(90:.3cm)] (135:.8cm);
	\draw (-45:.8cm) to [curve through=(-90:.3cm)] (-135:.8cm);
\end{tikzpicture}}

\newfig{\twostrandid}{\twostrandidbox}{\begin{tikzpicture}[baseline=-0.5ex,rotate=90,scale=0.8]
	\draw (45:.8cm) to [curve through=(90:.3cm)] (135:.8cm);
	\draw (-45:.8cm) to [curve through=(-90:.3cm)] (-135:.8cm);
\end{tikzpicture}}

\newfig{\symcross}{\symcrossbox}{\begin{tikzpicture}[baseline=-0.5ex,scale=0.8]
	\draw (45:.8cm) -- (-135:.8cm);
	\draw (-45:.8cm) -- (135:.8cm);
\end{tikzpicture}}

\newfig{\braidcross}{\braidcrossbox}{\begin{tikzpicture}[baseline=-0.5ex,scale=0.8]
	\draw (45:.8cm) -- (-135:.8cm);
	\draw[line width=1mm,white,double=black] (-45:.8cm) -- (135:.8cm);
\end{tikzpicture}}

\newfig{\invbraidcross}{\invbraidcrossbox}{\begin{tikzpicture}[baseline=-0.5ex,scale=0.8]
	\draw (-45:.8cm) -- (135:.8cm);
	\draw[line width=1mm,white,double=black] (45:.8cm) -- (-135:.8cm);
\end{tikzpicture}}


\newfig{\drawcrossX}{\drawcrossXbox}{ \begin{tikzpicture}[baseline=-0.5ex,scale=0.8,rotate=90]
 	\draw ([out angle=30].2,.3) .. (45:.8cm);
	\draw (.2,.3) -- (.2,-.3);
 	\draw ([out angle=-30].2,-.3) .. (-45:.8cm);
        \draw ([out angle=-150].2,-.3) ..([in angle=-70]135:.8cm);
 	\draw[line width=1mm,white]
              ([out angle=150].2,.3) .. ([in angle=70]-135:.8cm);
 	\draw ([out angle=150].2,.3) .. ([in angle=70]-135:.8cm);
\end{tikzpicture}}

\newfig{\drawinvcrossX}{\drawinvcrossXbox}{ \begin{tikzpicture}[baseline=-0.5ex,scale=0.8,rotate=90]
	\draw ([out angle=30].2,.3) .. (45:.8cm);
	\draw (.2,.3) -- (.2,-.3);
 	\draw ([out angle=-30].2,-.3) .. (-45:.8cm);
 	\draw ([out angle=150].2,.3) .. ([in angle=70]-135:.8cm);
        \draw[line width=1mm,white] ([out angle=-150].2,-.3) ..([in angle=-70]135:.8cm);
        \draw([out angle=-150].2,-.3) ..([in angle=-70]135:.8cm);
\end{tikzpicture}}


\newfig{\drawsymcrossX}{\drawsymcrossXbox}{ \begin{tikzpicture}[baseline=-0.5ex,scale=0.8,rotate=90]
 	\draw ([out angle=30].2,.3) .. (45:.8cm);
	\draw (.2,.3) -- (.2,-.3);
 	\draw ([out angle=-30].2,-.3) .. (-45:.8cm);
        \draw ([out angle=-150].2,-.3) ..([in angle=-70]135:.8cm);
 	\draw%[line width=1mm,white,double=black]
              ([out angle=150].2,.3) .. ([in angle=70]-135:.8cm);
\end{tikzpicture}}


\newfig{\twist}{\twistbox}{
  \begin{tikzpicture}[baseline=-0.5ex,scale=0.8]
    \draw[line width=1mm,white,double=black]
       ([out angle=-15]135:.8cm)..([blank=soft]-60:.4cm)..(-120:.4cm)..([in angle=-165]45:.8cm);
    \draw[line width=1mm,white,double=black,use previous Hobby path={invert soft blanks,disjoint}];
  \end{tikzpicture}}


\newfig{\symtwist}{\symtwistbox}{
  \begin{tikzpicture}[baseline=-0.5ex,scale=0.8]
    \draw
       ([out angle=-15]135:.8cm)..([blank=soft]-60:.4cm)..(-120:.4cm)..([in angle=-165]45:.8cm);
   \draw[use previous Hobby path={invert soft blanks,disjoint}];
  \end{tikzpicture}}


\newfig{\twistvertex}{\twistvertexbox}{
\begin{tikzpicture}[baseline=-0.5ex,scale=0.8]
  \draw (0,-0.4)--(0,-0.8);
  \draw ([out angle=-170]30:0.8cm)..([in angle=160](0,-0.4);
  \draw[line width=1mm,white,double=black]
        ([out angle=-10]150:0.8cm)..([in angle=20](0,-0.4);
\end{tikzpicture}}

\newfig{\symtwistvertex}{\symtwistvertexbox}{
\begin{tikzpicture}[baseline=-0.5ex,scale=0.8]
  \draw (0,-0.4)--(0,-0.8);
  \draw ([out angle=-170]30:0.8cm)..([in angle=160](0,-0.4);
  \draw%[line width=1mm,white,double=black]
        ([out angle=-10]150:0.8cm)..([in angle=20](0,-0.4);
\end{tikzpicture}}


\newfig{\loopvertex}{\loopvertexbox}{
\begin{tikzpicture}[baseline=-0.5ex,scale=0.8]
  \draw (0,-0.5)--(0,-0.8);
  \draw ([out angle=150]0,-0.5)..(-0.02,0.6)..(0.02,0.6)..([in angle=30]0,-0.5);
\end{tikzpicture}
}

\newfig{\idtangle}{\idtanglebox}{
\begin{tikzpicture}[baseline=-0.5ex,scale=0.8]
\draw (0,0) node {$D$};
\draw[densely dashed] (0,0) circle (.5cm);
\draw (45:.5cm) -- (45:0.8cm);
\draw (135:.5cm) -- (135:0.8cm);
\draw (-45:.5cm) -- (-45:0.8cm);
\draw (-135:.5cm) -- (-135:0.8cm);
\end{tikzpicture}}
    
\newfig{\Rcycle}{\Rcyclebox}{
\begin{tikzpicture}[baseline=-0.5ex,scale=0.8]
\draw (0,0) node {$D$};
\draw[densely dashed] (0,0) circle (.4cm);
\draw (45:.4cm) -- (45:1.1cm);
\draw ([out angle=-70]-45:.4cm) .. (-90:.7cm) .. ([in angle=0]-135:1.1cm);
\draw ([out angle=-160]-135:.4cm) .. (180:.7cm) .. ([in angle=-90]135:1.1cm);
\draw[line width=0.8mm,white,double=black]
      ([out angle=90]135:.4cm) .. (45:.7cm) .. ([in angle=90]-45:1.1cm);
\end{tikzpicture}}

\newfig{\ladder}{\ladderbox}{
\begin{tikzpicture}[baseline=-0.5ex,scale=0.8]
\draw (0,0) node {$D$};
\draw[densely dashed] (0,0) circle (.5cm);
\draw (135:.5cm) -- (135:0.9cm);
\draw (-135:.5cm) -- (-135:0.9cm);
\draw ([out angle=45]45:.5cm) .. ([in angle=150]0.8,0.45);
\draw ([out angle=-45]-45:.5cm) .. ([in angle=-150]0.8,-0.45);
\draw (0.8,0.45)--(0.8,-0.45);
\draw ([out angle=30]0.8,0.45) .. ([in angle=-135]1.2,0.636);
\draw ([out angle=-30]0.8,-0.45) .. ([in angle=135]1.2,-0.636);
\end{tikzpicture}}

\newfig{\pentagon}{\pentagonbox}{
\begin{tikzpicture}[scale=.12,baseline=-2]
\draw (36:1) -- (108:1) -- (180:1) -- (252:1) -- (-36:1) -- (36:1);
\end{tikzpicture}
}

\newfig{\psq}{\psqbox}{
\begin{tikzpicture}[scale=.15,baseline=-2]
\draw (36:1) -- (108:1) -- (180:1) -- (252:1) -- (-36:1) -- (36:1) -- +(1.2,0) -- ($(-36:1)+(1.2,0)$) -- (-36:1);
\end{tikzpicture}
}

\newfig{\sqsq}{\sqsqbox}{
\begin{tikzpicture}[scale=.15]
\draw (0,0) -- (2,0) -- (2,1) -- (0,1) -- cycle (1,0) -- (1,1);
\end{tikzpicture}
}

\newfig{\overviolin}{\overviolinbox}{
\begin{tikzpicture}[baseline=0]
	\coordinate (E1) at (45:.8cm);
	\coordinate (E2) at (225:.8cm);
	\coordinate (E3) at (270:.8cm);
	\coordinate (P1) at (0,0);
	\coordinate (P2) at (120:.6cm);
	\coordinate (P3) at (.4cm,.2cm);
	\coordinate (C1) at (100:.9cm);
	\coordinate (C2) at (135:.5cm);
	\coordinate (C3) at (.8,.1);
	\draw (E1) -- (P1) -- (E3);
	\draw (P1) .. controls  (C1) .. (P2);
	\draw[white, line width=4pt] (P2) .. controls  (C2) .. (P3);	
	\draw[white, line width=4pt] (P3) .. controls (C3) .. (E2);
	\draw (P2) .. controls  (C2) .. (P3);	
	\draw (P3) .. controls (C3) .. (E2);
%	\path (P1) ++(330:.15cm) node {$\bullet$};
\end{tikzpicture}
}

\newfig{\rotatedtrivalent}{\rotatedtrivalentbox}{
\begin{tikzpicture}[baseline=0]
	\coordinate (E1) at (45:.8cm);
	\coordinate (E2) at (225:.8cm);
	\coordinate (E3) at (270:.8cm);
	\coordinate (P0) at (0,0);
	\coordinate (C1) at (-.1,.7);
	\coordinate (C2) at (-.3,-.2);
	\coordinate (C3) at (.2,-.2);
	\draw (P0) .. controls (C1) .. (E2);
	\draw (P0) .. controls (C2) .. (E3);
	\draw (P0) .. controls (C3) .. (E1);
%	\path (P0) ++(270:.2cm) node {$\bullet$};
\end{tikzpicture}
}

\newcommand{\undercrossingfork}{
\begin{tikzpicture}[baseline=0]
	\coordinate (E1) at (90-72:.8cm);
	\coordinate (E2) at (90:.8cm);
	\coordinate (E3) at (90+72:.8cm);
	\coordinate (E4) at (90+2*72:.8cm);
	\coordinate (E5) at (90+3*72:.8cm);
	\coordinate (P1) at (270:.2cm);
	\draw (E4) -- (P1) -- (E5);
	\draw (P1) -- (E2);
	\draw[white, line width = 4pt] (E1) -- (E3);
	\draw (E1) -- (E3);
%	\path (P1) ++(270:.2cm) node {$\bullet$};
\end{tikzpicture}}

\newcommand{\undercrossingforktwoleft}{
\begin{tikzpicture}[baseline=0,rotate =144]
	\coordinate (E1) at (90-72:.8cm);
	\coordinate (E2) at (90:.8cm);
	\coordinate (E3) at (90+72:.8cm);
	\coordinate (E4) at (90+2*72:.8cm);
	\coordinate (E5) at (90+3*72:.8cm);
	\coordinate (P1) at (270:.2cm);
	\draw (E4) -- (P1) -- (E5);
	\draw (P1) -- (E2);
	\draw[white, line width = 4pt] (E1) -- (E3);
	\draw (E1) -- (E3);
%	\path (P1) ++(270:.2cm) node {$\bullet$};
\end{tikzpicture}}


\newcommand{\overcrossingfork}{
\begin{tikzpicture}[baseline=0]
	\coordinate (E1) at (90-72:.8cm);
	\coordinate (E2) at (90:.8cm);
	\coordinate (E3) at (90+72:.8cm);
	\coordinate (E4) at (90+2*72:.8cm);
	\coordinate (E5) at (90+3*72:.8cm);
	\coordinate (P1) at (270:.2cm);
	\draw (E4) -- (P1) -- (E5);
	\draw (E1) -- (E3);
	\draw[white, line width = 4pt] (P1) -- (E2);
	\draw (P1) -- (E2);
%	\path (P1) ++(270:.2cm) node {$\bullet$};
\end{tikzpicture}
}

\newcommand{\overcrossingforktworight}{
\begin{tikzpicture}[baseline=0, rotate=216]
	\coordinate (E1) at (90-72:.8cm);
	\coordinate (E2) at (90:.8cm);
	\coordinate (E3) at (90+72:.8cm);
	\coordinate (E4) at (90+2*72:.8cm);
	\coordinate (E5) at (90+3*72:.8cm);
	\coordinate (P1) at (270:.2cm);
	\draw (E4) -- (P1) -- (E5);
	\draw (E1) -- (E3);
	\draw[white, line width = 4pt] (P1) -- (E2);
	\draw (P1) -- (E2);
%	\path (P1) ++(270:.2cm) node {$\bullet$};
\end{tikzpicture}
}



\newcommand{\underunderfork}{
\begin{tikzpicture}[baseline=0]
	\coordinate (E1) at (90-72:.8cm);
	\coordinate (E2) at (90:.8cm);
	\coordinate (E3) at (90+72:.8cm);
	\coordinate (E4) at (90+2*72:.8cm);
	\coordinate (E5) at (90+3*72:.8cm);
	\coordinate (P1) at (90:.2cm);
	\draw (E4) .. controls (180:.2cm) .. (P1) .. controls (0:.2cm) ..  (E5);
	\draw (P1) -- (E2);
	\draw[white, line width = 4pt] (E1) .. controls (270:.4cm) .. (E3);
	\draw (E1) .. controls (270:.4cm) .. (E3);
\end{tikzpicture}}

\newcommand{\underoverfork}{
\begin{tikzpicture}[baseline=0]
	\coordinate (E1) at (90-72:.8cm);
	\coordinate (E2) at (90:.8cm);
	\coordinate (E3) at (90+72:.8cm);
	\coordinate (E4) at (90+2*72:.8cm);
	\coordinate (E5) at (90+3*72:.8cm);
	\coordinate (P1) at (90:.2cm);
	\coordinate (P2) at (90:.-.4cm);
	\draw (P1) .. controls (0:.2cm) ..  (E5);
	\draw (P1) -- (E2);
	\draw[white, line width = 4pt] (E1) .. controls (270:.4cm) .. (E3);
	\draw (E1) .. controls (270:.4cm) .. (E3);
	\draw[white, line width = 4pt] (E4) .. controls (180:.2cm) .. (P1);
	\draw (E4) .. controls (180:.2cm) .. (P1);
\end{tikzpicture}}

\newcommand{\overunderfork}{
\begin{tikzpicture}[baseline=0]
	\coordinate (E1) at (90-72:.8cm);
	\coordinate (E2) at (90:.8cm);
	\coordinate (E3) at (90+72:.8cm);
	\coordinate (E4) at (90+2*72:.8cm);
	\coordinate (E5) at (90+3*72:.8cm);
	\coordinate (P1) at (90:.2cm);
	\coordinate (P2) at (90:.-.4cm);
	\draw (E4) .. controls (180:.2cm) .. (P1);
	\draw (P1) -- (E2);
	\draw[white, line width = 4pt] (E1) .. controls (270:.4cm) .. (E3);
	\draw (E1) .. controls (270:.4cm) .. (E3);
	\draw[white, line width = 4pt] (P1) .. controls (0:.2cm) ..  (E5);
	\draw (P1) .. controls (0:.2cm) ..  (E5);	
\end{tikzpicture}}


\newcommand{\overoverfork}{
\begin{tikzpicture}[baseline=0]
	\coordinate (E1) at (90-72:.8cm);
	\coordinate (E2) at (90:.8cm);
	\coordinate (E3) at (90+72:.8cm);
	\coordinate (E4) at (90+2*72:.8cm);
	\coordinate (E5) at (90+3*72:.8cm);
	\coordinate (P1) at (90:.2cm);
	\draw (E1) .. controls (270:.4cm) .. (E3);
	\draw[white, line width = 4pt] (E4) .. controls (180:.2cm) .. (P1) .. controls (0:.2cm) ..  (E5);
	\draw (E4) .. controls (180:.2cm) .. (P1) .. controls (0:.2cm) ..  (E5);
	\draw (P1) -- (E2);
\end{tikzpicture}
}

\newcommand{\twisttreethreefour}{
\begin{tikzpicture}[baseline=0, use Hobby shortcut]
	\coordinate (E1) at (90-72:.8cm);
	\coordinate (E2) at (90:.8cm);
	\coordinate (E3) at (90+72:.8cm);
	\coordinate (E4) at (90+2*72:.8cm);
	\coordinate (E5) at (90+3*72:.8cm);
	\coordinate (F1) at (90-72:.5cm);
	\coordinate (F2) at (90:.5cm);
	\coordinate (F3) at (90+72-36:.3cm);
	\coordinate (F4) at (90+2*72+36:.3cm);
	\coordinate (F5) at (90+3*72:.5cm);
	\draw (E2) -- (F2);
	\draw (E1) -- (F1);
	\draw (E5) -- (F5);
	\draw (F3) .. (F2) .. (F1) .. (F5) .. (F4);
	\draw (E4) -- (F3);
	\draw[white, line width = 4pt] (E3) -- (F4);
	\draw (E3) -- (F4);
\end{tikzpicture}
}

\newcommand{\twisttreefiveone}{
\begin{tikzpicture}[baseline=0, use Hobby shortcut]
	\coordinate (E1) at (90-72:.8cm);
	\coordinate (E2) at (90:.8cm);
	\coordinate (E3) at (90+72:.8cm);
	\coordinate (E4) at (90+2*72:.8cm);
	\coordinate (E5) at (90+3*72:.8cm);
	\coordinate (F1) at (90-72+36:.3cm);
	\coordinate (F2) at (90:.5cm);
	\coordinate (F3) at (90+72:.5cm);
	\coordinate (F4) at (90+2*72:.5cm);
	\coordinate (F5) at (90+3*72-36:.3cm);
	\draw (E2) -- (F2);
	\draw (E3) -- (F3);
	\draw (E4) -- (F4);
	\draw (F1) .. (F2) .. (F3) .. (F4) .. (F5);
	\draw (E1) -- (F5);
	\draw[white, line width = 4pt] (E5) -- (F1);
	\draw (E5) -- (F1);
\end{tikzpicture}
}

